\documentclass[semifinal,survey]{cpecmu}
%% This is a sample document demonstrating how to use the CPECMU
%% project template. If you are having trouble, see "cpecmu.pdf" for
%% documentation.

\projectNo{S807-2}
\acadyear{2021}

\titleTH{โปรแกรมวางแผนปฏิทินการศึกษา มหาวิทยาลัยเชียงใหม่}
\titleEN{Chiang Mai University's academic calendar planner}

\author{นายเจษฎา จินะกะ}{Jetsada Jinaka}{620612144}
\author{นายเอื้อบุญ เรือนคำฟู}{Aueboon Ruanekamfu}{620612170}

\cpeadvisor{chinawat}
\cpecommittee{pruet}
\cpecommittee{lachana}

%% Some possible packages to include:
\usepackage[final]{graphicx} % for including graphics

%% Add bookmarks and hyperlinks in the document.
\PassOptionsToPackage{hyphens}{url}
\usepackage[colorlinks=true,allcolors=Blue4,citecolor=red,linktoc=all]{hyperref}
\def\UrlFont{\thaifonttt}

%% Set up commenting
\iffinal
  \usepackage[disabled]{authcomments}
\else
  \usepackage{authcomments}
\fi
\newcommenter{CI}{0.0,0.5625,0.0}  % green

%% Needed just by this example, but maybe not by most reports
\usepackage{afterpage} % for outputting
\usepackage{pdflscape} % for landscape figures and tables. 

%% Some other useful packages. Look these up to find out how to use
%% them.
% \usepackage{natbib}    % for author-year citation styles
% \usepackage{txfonts}
% \usepackage{appendix}  % for appendices on a per-chapter basis
% \usepackage{xtab}      % for tables that go over multiple pages
% \usepackage{subfigure} % for subfigures within a figure
% \usepackage{pstricks,pdftricks} % for access to special PostScript and PDF commands
% \usepackage{nomencl}   % if you have a list of abbreviations

%% if you're having problems with overfull boxes, you may need to increase
%% the tolerance to 9999
% \tolerance=9999

\bibliographystyle{plain}
% \bibliographystyle{IEEEbib}

% \renewcommand{\topfraction}{0.85}
% \renewcommand{\textfraction}{0.1}
% \renewcommand{\floatpagefraction}{0.75}

%% Example for glossary entry
%% Need to use glossary option
%% See glossaries package for complete documentation.
\ifglossary
  \newglossaryentry{lorem ipsum}{
    name=lorem ipsum,
    description={derived from Latin dolorem ipsum, translated as ``pain itself''}
  }
\fi

%% Uncomment this command to preview only specified LaTeX file(s)
%% imported with \include command below.
%% Any other file imported via \include but not specified here will not
%% be previewed.
%% Useful if your report is large, as you might not want to build
%% the entire file when editing a certain part of your report.
% \includeonly{chapters/intro,chapters/background}

\begin{document}
\maketitle
\makesignature

\ifproject
   \begin{abstractTH}
      \par ในแต่ละปีการศึกษา สำนักทะเบียนและประมวลผลมหาวิทยาลัยเชียงใหม่ จำเป็นจะต้องจะต้องจัดทำร่างปฏิทินการศึกษาสำหรับปีการศึกษาถัดไปเป็นประจำทุกปี \enskip
      ซึ่งในการทำร่างปฏิทิน ซึ่งในปฏิทินประกอบไปด้วยกำหนดการของกิจกรรมการศึกษาต่างๆ เช่น วันเปิดภาคเรียน วันลงทะเบียนเรียน วันสอบ เป็นต้น\enskip
      ซึ่งวันกำหนดการต่างๆ จะมีเงื่อนไขที่อ้างอิงกับวันกำหนดการต่างๆ ซึ่งทำให้การทำร่งปฏิทินขึ้นมาหนึ่งร่าง มีความยุ่งยากในการทำมากจนใช้เวลาเป็นสัปดาห์\enskip
      และยังต้องมาแก้ใหม่หากกรรมการของมหาวิทยาลัยเชียงใหม่ไม่ให้ความเห็นชอบ\enskip

      จากปัญหาข้างต้น ผู้จัดทำได้จัดทำโปรแกรมวางแผนปฏิทินการศึกษา มหาวิทยาลัยเชียงใหม่ เพื่อเป็นเครื่องมือที่ช่วยให้ผู้จัดทำร่างปฏิทินของสำนักทะเบียน
      เพื่อช่วยลดระยะในการจัดทำปฏิทินลงได้ อีกทั้งสามารถเพิ่มวามสะดวกในการจัดทำร่างปฏิทินได้ สามารถแก้ไข สร้าง หรือทำซ้ำร่างปฏิทินที่มีฉบับที่คล้าคลึงกัน
      โปรแกรมดังกล่าวก็สามารถตอบสนองได้
   \end{abstractTH}

   \begin{abstract}
      in each academic year Chiang Mai University Registration and Processing Office It is necessary to make
      Outline an academic calendar for the next academic year annually. which in drafting the calendar in which the accompanying calendar
      along with the schedule of various educational activities such as the opening day of the semester registration day, exam date, etc.
      which scheduled dates There will be conditions that refer to various scheduled dates. which makes one draft of the calendar have
      The complexity of making it took weeks. And still have to come and fix it again if the director of the University of Chiang-
      new disapproval
      from the above problem The organizer has prepared an academic calendar planning program. Chiang Mai University for
      A tool that helps drafters of the registry's calendar. To help reduce the time for making calendars, it can also be
      It is convenient to create a draft calendar. You can edit, create or duplicate a draft calendar with similar editions.
      together, such programs can respond.
   \end{abstract}

   \iffalse
      \begin{dedication}
         This document is dedicated to all Chiang Mai University students.

         Dedication page is optional.
      \end{dedication}
   \fi % \iffalse

   \begin{acknowledgments}
      Your acknowledgments go here. Make sure it sits inside the
      \texttt{acknowledgment} environment.

      \acksign{2022}{2}{18}
   \end{acknowledgments}%
\fi % \ifproject

\contentspage

\ifproject
   \figurelistpage % สารบัญรูป 

   \tablelistpage  % สารบัญตาราง
\fi % \ifproject

% \abbrlist % this page is optional

% \symlist % this page is optional

% \preface % this section is optional


\pagestyle{empty}\cleardoublepage
\normalspacing \setcounter{page}{1} \pagenumbering{arabic} \pagestyle{cpecmu}

\chapter{\ifenglish Introduction\else บทนำ\fi}

\section{\ifenglish Project rationale\else ที่มาของโครงงาน\fi}
ในแต่ละปีการศึกษา สำนักทะเบียนและประมวลผล มหาวิทยาลัยเชียงใหม่ จำเป็นจะต้องจัดทำร่างปฏิทินการศึกษาสำหรับปีการศึกษาถัดไป เพื่อให้กรรมการบริหารมหาวิทยาลัยอนุมัติล่วงหน้า

ปฏิทินการศึกษาประกอบไปด้วยกำหนดการของกิจกรรมการศึกษาต่างๆ เช่น วันเปิดภาคเรียน วันลงทะเบียนเรียน วันสุดท้ายของการถอนกระบวนวิชา และวันสอบ เป็นต้น
%
กิจกรรมการศึกษาต่างๆ ส่วนใหญ่นั้นจะถูกกำหนดเป็นเงื่อนไขที่อ้างอิงกับวันเปิดภาคเรียน เช่น วันสุดท้ายของการเรียนการสอน จะถัดจากวันเปิดภาคเรียนประมาณ 16 สัปดาห์ จากนั้น จะเป็นการสอบปลายภาค ระยะเวลา 2 สัปดาห์ แล้วตามด้วยวันประกาศผลการศึกษา หลังจากสอบปลายภาควันสุดท้ายไปประมาณ 2 สัปดาห์
%
จะเห็นว่า หากกำหนดวันเปิดภาคการศึกษาให้ชัดเจนแล้ว กิจกรรมอื่นๆ จะสามารถจัดวางได้โดยอัตโนมัติ จึงทำให้การร่างปฏิทินการศึกษานั้นไม่ควรใช้เวลามากนัก

แต่ในความเป็นจริงแล้ว สำนักทะเบียนและประมวลผลยังขาดเครื่องมือที่จะอำนวยความสะดวกในการร่างปฏิทินการศึกษา ทำให้ต้องใช้เวลาในการสร้างแต่ละร่างถึง 2 สัปดาห์เป็นอย่างต่ำ
%
สาเหตุหลักๆ ในความล่าช้าดังกล่าว คือเงื่อนไขสำหรับกิจกรรมการศึกษาต่างๆ ที่ไม่ได้ระบุไว้เป็นลายลักษณ์อักษรให้ชัดเจน เพื่อที่จะสามารถนำมาใช้ซ้ำได้ ทำให้ผู้จัดทำร่างปฏิทินต้องกำหนดเงื่อนไขดังกล่าวในทุกๆ ปี ก่อนจะวางโครงร่างปฏิทินโดยการนับวันด้วยมือ
%
นอกจากนี้ หากกรรมการบริหารมหาวิทยาลัยมีมติให้แก้ไขร่างดังกล่าว ซึ่งอาจจะเกิดขึ้นได้หลายครั้งในแต่ละปีการศึกษา จะทำให้ผู้จัดทำร่างปฏิทินเสียเวลาเพิ่มเติมมากกว่าที่ควรจะเป็น เนื่องจากจะต้องเริ่มกระบวนการร่างปฏิทินใหม่ทั้งหมด
%
รวมไปปัญหาการที่จะแก้ไขวันหรือกิจกรรมในปฏิทินอย่างกระทัน ก็ยังไม่มีเครื่องมือที่จะช่วยแก้ปัญหาได้ หากกรรมการบริหารมหาวิทยาลัยเชียงใหม่ต้องการที่จะเปรียบเทียบหรือดูตัวอย่างของปฏิทินปีก่อนๆ แล้วนำมาแก้ไขเพียงบางส่วน ผู้จัดทำร่างปฏิทินก็จะต้องเตรียมเอกสารปฏิทินมาหลายฉบับ ซึ่งทำให้เกิดปัญหาให้แก่ผู้จัดทำร่างปฏิทินเป็นอย่างมาก

จากปัญหาการสร้างปฏิทินการศึกษาข้างต้น ผู้จัดทำโครงงานจึงมีแนวคิดที่จะสร้างโปรแกรมวางแผนปฏิทินการศึกษา มหาวิทยาลัยเชียงใหม่  
%
เพื่อที่จะเป็นเครื่องมือที่จะช่วยให้ผู้จัดทำร่างปฏิทินของสำนักทะเบียนสามารถลดเวลาในการจัดทำปฏิทินลงได้
%
อีกทั้งยังสามารถเพิ่มความสะดวกในการจัดทำร่างปฏิทิน ไม่ว่าจะเป็นการเปลี่ยนแปลงวันกิจกรรมได้อย่างรวดเร็ว การแก้ไขร่างปฏิทินอย่างกระทันหัน การสร้างร่างปฏิทินหลายฉบับที่มีความคล้ายคลึงกัน โปรแกรมดังกล่าวก็สามารถตอบสนองได้ 

%     ในปัจจุบันปฏิทินการศึกษาของมหาวิทยาลัยเชียงใหม่นั้นจัดทำโดยทางสำนักทะเบียนของมหาวิทยาลัยเชียงใหม่ 
% โดยที่ในการที่จะสร้างปฏิทินขึ้นมาได้นั้น จะต้องมีการร่างโครงร่างของปฏิทินออกมา โดยที่ในการที่จะร่างปฏิทินนั้ก็จะต้องมีเงื่อนไขในการส้รางปฏิทินต่างๆ ไม่ว่าจะเป็น การกำหนดวันเปิดภาคเรียน การกำหนดระยะเวลาของคาบเรียน วันปิดภาคเรียน ไปจนถึงวันลงทะเบียน 
% เราได้พบปัญหาว่าใน 1 ปฏิทินการศึกษานั้น ใช้เวลาในการร่างปฏิทินไม่ต่ำกว่า 2 สัปดาห์ เนื่องจากทางผู้จัดทำต้องกำหนดเงื่อนไขที่ได้กล่าวข้างต้น ถึงจะทำการวางโครงร่างของปฏิทินได้
% และเมื่อสามารถวางโครงร่างได้แล้วก็จะต้องนำโครงร่างไปส่งให้เป็นที่พิจารณาแก่คณะกรรมการบริหารของมหาวิทยาลัยเชียงใหม่ ซึ่งจะทำให้ต้องมีการแก้ไขอยู่หลายๆครั้ง จึงทำให้การทำปฏิทินวันเปิดภาคเรียนนั้นใช้เวลานานมากเกินไป เราจึงได้เล็งปัญหาของการสร้างปฏทินการศึกษา
% ซึ่งจัดทำโดยสำนักทะเบียนมหาวิทยาลัยเชียงใหม่ จึงได้เกิดเป็นโปรแกรมวางแผนปฏิทินการศึกษา มหาวิทยาลัยเชียงใหม่


\section{\ifenglish Objectives\else วัตถุประสงค์ของโครงงาน\fi}
\begin{enumerate}
    \item เพื่อลดเวลาในการจัดทำปฏิทินการศึกษา
    \item เพื่อสร้างระบบที่สามารถระบุเงื่อนไขต่างๆ ที่จำเป็นต่อการสร้างปฏิทินการศึกษาและสามารถแก้ไขได้ตามความต้องการ
    \item เพื่อสร้างระบบที่สามารถคัดลอกและทำซ้ำของปฏิทินได้ เมื่อต้องการที่จะแก้ปฏิทินหลายๆ ฉบับ และต้องเปลี่ยนการเปลี่ยนแปลงเพียงบางส่วน
\end{enumerate}

\section{\ifenglish Project scope\else ขอบเขตของโครงงาน\fi}

\subsection{\ifenglish Hardware scope\else ขอบเขตด้านฮาร์ดแวร์\fi}
\begin{enumerate}
\item โปรแกรมวางแผนปฏิทินการศึกษานี้สามารถใช้งานได้กับทุกอุปกรณ์ที่เข้าถึง web browser ได้ 
\end{enumerate}

\subsection{\ifenglish Software scope\else ขอบเขตด้านซอฟต์แวร์\fi}
\begin{enumerate}
\item โปรแกรมวางแผนเป็นโปรแกรมนี้จะเพิ่มวันหยุดและกิจกรรมมาให้โดยอัตโนมัติ แต่กิจกรรมที่นักศึกษาเป็นฝ่ายจัดจะไม่นับลงไปด้วย เช่น กิจกรรม Sports Day กิจกรรม Freshy Night เป็นต้น
\item ในการนำออกไฟล์ของโปรแกรมปฏิทิทินการศึกษานี้จะนำออกไฟล์มาเป็นไฟล์ .pdf และ ไฟล์สกุลของโปรแกรม Excel 
\item โปรแกรมวางแผนปฏิทินการศึกษานี้สามารถเข้าถึงได้เฉพาะบุคลากรของสำนักทะเบียน \\ มหาวิทยาลัยเชียงใหม่ที่มีชื่ออยู่ในระบบของ CMU Account เท่านั้น   
\end{enumerate}

\section{\ifenglish Expected outcomes\else ประโยชน์ที่ได้รับ\fi}
\begin{enumerate}
\item สามารถลดเวลาในการร่างปฏิทินการศึกษาให้ใช้เวลาในการทำลดลง
\item สามารถแก้ไขปฏิทินในที่ประชุมได้สะดวกขึ้นหากต้องการแก้กระทันหัน
\item โปรแกรมวางแผนปฏิทินการศึกษานี้สามารถใช้ได้จริง และเป็นประโยชน์ในการออกปฏิทินของสำนักทะเบียนและประมวลผล มหาวิทยาลัยเชียงใหม่
\end{enumerate}

\section{\ifenglish Technology and tools\else เทคโนโลยีและเครื่องมือที่ใช้\fi}

\subsection{\ifenglish Software technology\else เทคโนโลยีด้านซอฟต์แวร์\fi}
\begin{enumerate}
\item ใช้ Figma ในการออกแบบ
\item HTML เป็นภาษาที่ใข้ในการเขียนเว็บ 
\item ในส่วนของ front-end ใช้ React Js 
\item NodeJs ใช้ในการสร้าง web application ในส่วน back-end 
\item MongoDB เป็นเครื่องมือที่ใช้ในการจัดเก็บฐานข้อมูล
\end{enumerate}
\CIreply{wording style between items not parallel}



\pagebreak
\section{\ifenglish Project plan\else แผนการดำเนินงาน\fi}
\begin{plan}{1}{2022}{4}{2023}
    \planitem{1}{2022}{2}{2022}{ออกแบบ user interface และ user experience}
    \planitem{2}{2022}{3}{2022}{ออกแบบระบบฐานข้อมูล}
    \planitem{4}{2022}{1}{2023}{พัฒนาระบบ front-end และ back-end}
    \planitem{1}{2023}{3}{2023}{ทดลองระบบ}
    \planitem{4}{2023}{4}{2023}{นำเสนอและสรุปผลของการพัฒนาโปรแกรม}
\end{plan}

\section{\ifenglish Roles and responsibilities\else บทบาทและความรับผิดชอบ\fi}
บทบาทในส่วนของ web application มีการแบ่งออกเป็นสองฝั่ง ได้แก่ 
%
ฝั่งของหน้าบ้าน (front-end) ซึ่งเป็นฝั่งที่จำเป็นจะต้องรู้ในเรื่องของ HTML, CSS, JS มีความใจในส่วนของ UX/UI เพื่อออกแบบให้ผู้ใช้งานสามารถเข้าใจได้ง่าย 
%
รวมไปถึงการส่ง requests ส่งไปฝั่ง back-end ผ่าน API ที่จะต้องมีการรับ requests จาก front-end เช่น การดึงข้อมูลมาแสดงผล (GET) การแก้ไขข้อมูลบนฐานข้อมูล (PUT)  
ในส่วนของฝั่งหลังบ้าน (back-end) จำเป็นต้องจัดการในส่วนของฐานข้อมูลด้วย
%


ในช่วงแรกของการทำส่วนของ front-end UX/UI design นายเอื้อบุญ เรือนคำฟู เป็นผู้รับผิดชอบ  
%
ส่วนในฝั่งของ back-end นายเจษฎา จินะกะ เป็นผู้รับผิดชอบ

\section{\ifenglish%
Impacts of this project on society, health, safety, legal, and cultural issues
\else%
ผลกระทบด้านสังคม สุขภาพ ความปลอดภัย กฎหมาย และวัฒนธรรม
\fi}

\subsection{นักศึกษา}
    จากจุดประสงค์ของโครงงาน ผู้พัฒนาต้องการลดเวลาในการสร้างร่างปฏิทินการศึกษา โดยนักศึกษาจะได้รับผลกระทบ ซึ่งผลกระทบ
%
นั้นเกิดจากการลดเวลาในการร่างปฏิทินการศึกษา หากมหาวิทยาลัยมีการประกาศกำหนดการที่เร็วขึ้น นักศึกษาจะสามารถทราบกำหนดการและสามารถจัดเวลาเรียนของตัวเองได้อย่างเหมาะสม 
%

\subsection{มหาวิทยาลัยเชียงใหม่}
สืบเนื่องมาจากการทำให้การร่างปฏิทินการศึกษาง่ายต่อการแก้ไข ทำให้การประกาศการเปลี่ยนแปลงวันหยุดราชการต่างๆ หรือกำหนดการต่างๆ 
%
(ตัวอย่างเช่น กำหนดการณ์สอบ O-Net หรือการสอบต่างๆ ที่สำคัญ ที่สามารถเปลี่ยนแปลงได้ตลอดเวลา การสอบข้างต้นมีผลต่อการเปิดเทอมของมหาวิทยาลัย เนื่องจากเป็นเกณฑ์ในการรับนักศึกษาชั้นปีที่ 1 เข้ามาศึกษาในมหาวิทยาลัย) 
%
ที่เป็นเรื่องที่ต้องทำการแก้ไขในกำหนดการของร่างปฏิทิน จะสามารถนำมาแก้ไขในปฏิทินการศึกษาที่ทำการร่างไว้ได้สะดวกและประหยัดเวลาได้มากขึ้น

\subsection{สำนักทะเบียนและประมวลผล มหาวิทยาลัยเชียงใหม่}
จากที่ได้ติดต่อกับบุคลากรจากสำนักทะเบียนและประมวลผล มหาวิทยาลัยเชียงใหม่ พบว่าในการเริ่มทำปฏิทินเกิดจากการทำมือ และทำจากโปรแกรมคอมพิวเตอร์ต่างๆ ที่เป็นโปรแกรมจัดทำเอกสาร
%
หรือโปรแกรมสำหรับเอกสาร มาร่างกำหนดการของปฏิทินการศึกษาและใช้เวลาทำนานนับเดือน นอกจากนั้น หากนำไปเสนอในที่ประชุมจะต้องทำการแก้ไขในหลายๆ ร่าง จึงทำให้เสียเวลาในการทำส่วนอื่น
%
โครงงานนี้จึงจัดทำมาเพื่อสำนักทะเบียนและประมวลผล มหาวิทยาลัยเชียงใหม่ ซึ่งทำให้สะดวกต่อการแก้ไขร่างมากยิ่งขึ้น และประหยัดเวลาในการร่างปฏิทินอีกด้วย โดยฟังก์ชันการวางร่างปฏิทินอัตโนมัติ ทำให้ผู้จัดทำไม่ต้องมานั่งวางทีละวัน แต่จะเป็นการวางวันโดยอัตโนมัติแทน และง่ายต่อการทำซ้ำหรือแก้ไขหลายๆ ร่าง

\chapter{\ifenglish Background Knowledge and Theory\else ทฤษฎีที่เกี่ยวข้อง\fi}

การทำโครงงาน เริ่มต้นด้วยการศึกษาค้นคว้า ทฤษฎีที่เกี่ยวข้อง หรือ งานวิจัย/โครงงาน ที่เคยมีผู้นำเสนอไว้แล้ว ซึ่งเนื้อหาในบทนี้ก็จะเกี่ยวกับการอธิบายถึงสิ่งที่เกี่ยวข้องกับโครงงาน เพื่อให้ผู้อ่านเข้าใจเนื้อหาในบทถัดๆ ไปได้ง่ายขึ้น

\section{หน้าเว็บ}
หน้าเว็บ คือ หน้าเอกสารที่ถูกแสดงโดย เว็บเบราว์เซอร์ เพ่ือแสดงข้อมูลต่างๆ ที่เป็นข้อความ รูปภาพ และส่ือผสมต่างๆ ซึ่งเนื้อหาของหน้าเว็บเป็นอย่างไร ขึ้นอยู่กับวัตถุประสงค์ของ เจ้าของหน้าเว็บ ไม่ว่าจะเป็นเนื้อหาเกี่ยวกับการศึกษาธุรกิจ หรือ ความบันเทิง เป็นต้น


\section{ HTML}
HTML ย่อมาจาก HyperText Markup Language เป็น ภาษาคอมพิวเตอร์ท่ีใช้สร้างหน้าเว็บในรูปแบบของ ไฟล์ HTML (คือไฟล์ที่มีนามสกุลเป็น .htm หรือ .html) ซึ่งมีเว็บเบราว์เซอร์เป็นโปรแกรมที่ใช้แปลงไฟล์ HTML เพื่อ แสดงผลในรูปของหน้าเว็บ
ไฟล์ HTML เป็นไฟล์รหัสแอสกี (ASCII) ถูกบันทึกในรูปของไฟล์เอกสาร (Text File) ที่สามารถถูกสร้างจากโปรแกรมสร้างไฟล์ ข้อความ เช่น Notepad หรือ Word Processing ทั่วๆ ไป ซึ่งลักษณะของไฟล์ HTML ประกอบไปด้วยแท็กต่างๆ ที่เป็นคําาส่ังของ HTML ซึ่งแท็กจะอยู่ภายในเครื่องหมาย <และ>

\section{ CSS}
CSS ย่อมาจาก Cascading Style Sheet  มักเรียกโดยย่อว่า "สไตล์ชีต" คือภาษาที่ใช้เป็นส่วนของการจัดรูปแบบการแสดงผลเอกสาร  HTML โดยที่ CSS กำหนดกฏเกณฑ์ในการระบุรูปแบบ (หรือ "Style") ของเนื้อหาในเอกสาร 
อันได้แก่ สีของข้อความ สีพื้นหลัง ประเภทตัวอักษร และการจัดวางข้อความ ซึ่งการกำหนดรูปแบบ หรือ Style นี้ใช้หลักการของการแยกเนื้อหาเอกสาร HTML ออกจากคำสั่งที่ใช้ในการจัดรูปแบบการแสดงผล กำหนดให้รูปแบบของการแสดงผลเอกสาร 
ไม่ขึ้นอยู่กับเนื้อหาของเอกสาร เพื่อให้ง่ายต่อการจัดรูปแบบการแสดงผลลัพธ์ของเอกสาร HTML

\section{JavaScript}
JavaScript หรือย่อด้วย JS เป็นภาษาเขียนโปรแกรมที่ถูกพัฒนาและปฏิบัติตามข้อกำหนดมาตรฐานของ ECMAScript
ภาษา JavaScript นั้นเป็นภาษาระดับสูง คอมไพล์ในขณะที่โปรแกรมรัน (JIT) และเป็นภาษาเขียนโปรแกรมแบบหลายกระบวนทัศน์ เช่น การเขียนโปรแกรมเชิงขั้นตอน การเขียนโปรแกรมเชิงวัตถุ หรือการเขียนโปรแกรมแบบ Functional; 
ภาษา JavaScript มีไวยากรณ์ที่เหมือนกับภาษา C ใช้วงเล็บเพื่อกำหนดบล็อคของคำสั่ง นอกจากนี้ JavaScript ยังเป็นภาษาที่มีประเภทข้อมูลแบบไดนามิกส์ เป็นภาษาแบบ Prototype-based และ First-class function

\section{Node.JS}
Node.js คือสภาพแวดล้อมการทำงานของภาษา JavaScript นอกเว็บเบราว์เซอร์ที่ทำงานด้วย V8 engine นั่นคือสามารถใช้ Node.js ในการพัตนาแอพพลิเคชันแบบ Command line แอพพลิเคชัน Desktop หรือแม้แต่เว็บเซิร์ฟเวอร์ได้ 
โดยที่ Node.js จะมี APIs ที่เราสามารถใช้สำหรับทำงานกับระบบปฏิบัติการ เช่น การรับค่าและการแสดงผล การอ่านเขียนไฟล์ และการทำงานกับเน็ตเวิร์ก เป็นต้น

\section{React.JS}
React เป็น JavaScript library ที่ใช้สำหรับสร้าง user interface ในฝั่งด้าน Front-end ที่ให้เราสามารถเขียนโค้ดในการสร้าง UI ที่มีความซับซ้อนแบ่งเป็นส่วนเล็กๆออกจากกันได้ ซึ่งแต่ละส่วนสามารถแยกการทำงานออกจากกันได้อย่างอิสระ 
และทำให้สามารถนำชิ้นส่วน UI เหล่านั้นไปใช้ซ้ำได้อีก

\section{MongoDB}
MongoDB เป็น open-source document database โดยเป็นฐานข้อมูลแบบ NoSQL หรือเรียกว่าไม่มีความสัมพันธ์ของตารางแบบ SQL ทั่วๆไป แต่จะเก็บข้อมูลเป็นแบบ JSON แทน การบันทึกข้อมูลทุกๆ record ใน MongoDB 
เราจะเรียกมันว่า Documentซึ่งจะเก็บค่าเป็น key และ value หรือก็คือไฟล์ JSON

\section{JSON}
ย่อมาจาก JavaScript Object Notation เป็นฟอร์แมตสำหรับแลกเปลี่ยนข้อมูลคอมพิวเตอร์ \\ ฟอร์แมต JSON นั้นอยู่ในรูปข้อความธรรมดา (plain text) ที่ทั้งมนุษย์และโปรแกรมคอมพิวเตอร์สามารถอ่านเข้าใจได้ 
มาตรฐานของฟอร์แมต JSON คือ RFC 4627 มี Internet media type เป็น application/json และมีนามสกุลของไฟล์เป็น .json
ปัจจุบัน JSON นิยมใช้ในเว็บแอปพลิเคชัน โดยเฉพาะ AJAX โดย JSON เป็นฟอร์แมตทางเลือกในการส่งข้อมูล นอกเหนือไปจาก XML ซึ่งนิยมใช้กันอยู่แต่เดิม สาเหตุที่ JSON เริ่มได้รับความนิยมเป็นเพราะกระชับและเข้าใจง่ายกว่า XML

JSON นั้นใช้ความสัมพันธ์ของภาษาจาวาสคริปต์ แต่ไม่ถูกมองว่าเป็นภาษาโปรแกรม กลับถูกมองว่าเป็นภาษาในการแลกเปลี่ยนข้อมูลมากกว่า ในปัจจุบันมีไลบรารีของภาษาโปรแกรมอื่นๆ ที่ใช้ประมวลผลข้อมูลในรูปแบบ JSON มากมาย







%\section{Third section}
%Section 3 text. The dielectric constant\index{dielectric constant}
%at the air-metal interface determines
%the resonance shift\index{resonance shift} as absorption or capture occurs
%is shown in Equation~\eqref{eq:dielectric}:

%\begin{equation}\label{eq:dielectric}
%k_1=\frac{\omega}{c({1/\varepsilon_m + 1/\varepsilon_i})^{1/2}}=k_2=\frac{\omega
%\sin(\theta)\varepsilon_\mathit{air}^{1/2}}{c}
%\end{equation}

%\noindent
%where $\omega$ is the frequency of the plasmon, $c$ is the speed of
%light, $\varepsilon_m$ is the dielectric constant of the metal,
%$\varepsilon_i$ is the dielectric constant of neighboring insulator,
%and $\varepsilon_\mathit{air}$ is the dielectric constant of air.

%\section{About using figures in your report}

% define a command that produces some filler text, the lorem ipsum.
%\newcommand{\loremipsum}{
%  \textit{Lorem ipsum dolor sit amet, consectetur adipisicing elit, sed do
%  eiusmod tempor incididunt ut labore et dolore magna aliqua. Ut enim ad
%  minim veniam, quis nostrud exercitation ullamco laboris nisi ut
%  aliquip ex ea commodo consequat. Duis aute irure dolor in
%  reprehenderit in voluptate velit esse cillum dolore eu fugiat nulla
%  pariatur. Excepteur sint occaecat cupidatat non proident, sunt in
%  culpa qui officia deserunt mollit anim id est laborum.}\par}

%\begin{figure}
%  \centering

%  \fbox{
%     \parbox{.6\textwidth}{\loremipsum}
%  }

  % To include an image in the figure, say myimage.pdf, you could use
  % the following code. Look up the documentation for the package
  % graphicx for more information.
  % \includegraphics[width=\textwidth]{myimage}

%  \caption[Sample figure]{This figure is a sample containing \gls{lorem ipsum},
%  showing you how you can include figures and glossary in your report.
%  You can specify a shorter caption that will appear in the List of Figures.}
%  \label{fig:sample-figure}
%\end{figure}

%Using \verb.\label. and \verb.\ref. commands allows us to refer to
%figures easily. If we can refer to Figures
%\ref{fig:walrus} and \ref{fig:sample-figure} by name in the {\LaTeX}
%source code, then we will not need to update the code that refers to it
%even if the placement or ordering of the figures changes.

%\loremipsum\loremipsum

% This code demonstrates how to get a landscape table or figure. It
% uses the package lscape to turn everything but the page number into
% landscape orientation. Everything should be included within an
% \afterpage{ .... } to avoid causing a page break too early.
%\afterpage{
%  \begin{landscape}
%  \begin{table}
%    \caption{Sample landscape table}
%    \label{tab:sample-table}

%   \centering

%    \begin{tabular}{c||c|c}
%        Year & A & B \\
%        \hline\hline
%        1989 & 12 & 23 \\
%        1990 & 4 & 9 \\
%        1991 & 3 & 6 \\
%    \end{tabular}
%  \end{table}
%  \end{landscape}
%}

%\loremipsum\loremipsum\loremipsum

%\section{Overfull hbox}

%When the \verb.semifinal. option is passed to the \verb.cpecmu. document class,
%any line that is longer than the line width, i.e., an overfull hbox, will be
%highlighted with a black solid rule:
%\begin{center}
%\begin{minipage}{2em}
%juxtaposition
%\end{minipage}
%\end{center}

%\section{\ifenglish%
%\ifcpe CPE \else ISNE \fi knowledge used, applied, or integrated in this project
%\else%
%ความรู้ตามหลักสูตรซึ่งถูกนำมาใช้หรือบูรณาการในโครงงาน
%\fi
%}

%อธิบายถึงความรู้ และแนวทางการนำความรู้ต่างๆ ที่ได้เรียนตามหลักสูตร ซึ่งถูกนำมาใช้ในโครงงาน

%\section{\ifenglish%
%Extracurricular knowledge used, applied, or integrated in this project
%\else%
%ความรู้นอกหลักสูตรซึ่งถูกนำมาใช้หรือบูรณาการในโครงงาน
%\fi
%}

%อธิบายถึงความรู้ต่างๆ ที่เรียนรู้ด้วยตนเอง และแนวทางการนำความรู้เหล่านั้นมาใช้ในโครงงาน

\chapter{\ifproject%
\ifenglish Project Structure and Methodology\else โครงสร้างและขั้นตอนการทำงาน\fi
\else%
\ifenglish Project Structure\else โครงสร้างของโครงงาน\fi
\fi
}

ในบทนี้จะกล่าวถึงหลักการ และการออกแบบระบบไปจนถึงขั้นตอนการออกแบบจากความต้องการของผู้ใช้งาน

\makeatletter

% \renewcommand\section{\@startsection {section}{1}{\z@}%
%                                    {13.5ex \@plus -1ex \@minus -.2ex}%
%                                    {2.3ex \@plus.2ex}%
%                                    {\normalfont\large\bfseries}}

\makeatother
%\vspace{2ex}
% \titleformat{\section}{\normalfont\bfseries}{\thesection}{1em}{}
% \titlespacing*{\section}{0pt}{10ex}{0pt}

\section{การติดต่อและคุยงานเพื่อสรุปความต้องการของสำนักทะเบียน}

\begin{figure}
\begin{center}
% \includegraphics{800px-Briny_Beach.jpg}
\end{center}
% \caption[Poem]{The Walrus and the Carpenter}
\label{fig:walrus}
\end{figure}

% \subsection{The Black Kitten}
  เนื่องจากจุดประสงค์ของโครงงานนี้คือต้องการพัฒนาเว็บไซต์ให้แก่สำนักทะเบียน
  จึงจะต้องเริ่มจากการพูดคุยกับบุคลากรของสำนักทะเบียนเพื่อให้ได้
  ความต้องการที่แก้จริงของโครงงานโดยในปฏิทินจะมีเงื่อนไขต่างๆ อันสรุปได้ดังนี้

  The way Dinah washed her children's faces was this:  first she
held the poor thing down by its ear with one paw, and then with
the other paw she rubbed its face all over, the wrong way,
beginning at the nose:  and just now, as I said, she was hard at
work on the white kitten, which was lying quite still and trying
to purr---no doubt feeling that it was all meant for its good.

หลังจากที่ไดเงื่อนไขทั้งหมดครบแล้ว จะนำเงื่อนไขเหล่านี้มาแยกออกจากระบบที่ตอบสนอง
แบะเพิ่มความสะดวกสบายของผู้ใช้

% \subsection{The Reproach}

%   `Oh, you wicked little thing!' cried Alice, catching up the
% kitten, and giving it a little kiss to make it understand that it
% was in disgrace.  `Really, Dinah ought to have taught you better
% manners!  You OUGHT, Dinah, you know you ought!' she added,
% looking reproachfully at the old cat, and speaking in as cross a
% voice as she could manage---and then she scrambled back into the
% arm-chair, taking the kitten and the worsted with her, and began
% winding up the ball again.  But she didn't get on very fast, as
% she was talking all the time, sometimes to the kitten, and
% sometimes to herself.  Kitty sat very demurely on her knee,
% pretending to watch the progress of the winding, and now and then
% putting out one paw and gently touching the ball, as if it would
% be glad to help, if it might.

%   `Do you know what to-morrow is, Kitty?' Alice began.  `You'd
% have guessed if you'd been up in the window with me---only Dinah
% was making you tidy, so you couldn't.  I was watching the boys
% getting in stick for the bonfire---and it wants plenty of
% sticks, Kitty!  Only it got so cold, and it snowed so, they had
% to leave off.  Never mind, Kitty, we'll go and see the bonfire
% to-morrow.'  Here Alice wound two or three turns of the worsted
% round the kitten's neck, just to see how it would look:  this led
% to a scramble, in which the ball rolled down upon the floor, and
% yards and yards of it got unwound again.

%   `Do you know, I was so angry, Kitty,' Alice went on as soon as
% they were comfortably settled again, `when I saw all the mischief
% you had been doing, I was very nearly opening the window, and
% putting you out into the snow!  And you'd have deserved it, you
% little mischievous darling!  What have you got to say for
% yourself?  Now don't interrupt me!' she went on, holding up one
% finger.  `I'm going to tell you all your faults.  Number one:
% you squeaked twice while Dinah was washing your face this
% morning.  Now you can't deny it, Kitty:  I heard you!  What that
% you say?' (pretending that the kitten was speaking.)  `Her paw
% went into your eye?  Well, that's YOUR fault, for keeping your
% eyes open---if you'd shut them tight up, it wouldn't have
% happened.  Now don't make any more excuses, but listen!  Number
% two:  you pulled Snowdrop away by the tail just as I had put down
% the saucer of milk before her!  What, you were thirsty, were you?

\section{โครงสร้างของโครงงาน และการทำงานของโปแกรม}
\subsection{การทำงานของโปรแกรม}
จากการสรุปความต้องการของสำนักทะเบียนผ่านทางบุคลากรจึงสรุปออกมาเป็น User Flowหรือ
สิ่งที่แสดงเส้นทางของผู้ใช้แอพพลิเคชั่นได้ดังนี้
% \includegraphics{pic3.1.jpg}
% \caption[Poem]{รูปที่3.1 การทำงานเมื่อผู้ใช้ต้องการสร้างแบบปฏิทินใหม่}

จากรูปที่3.1 ผู้ใช้จะเริ่มจากการเข้าสู่ระบบโดยใช้ CMU QAuthหลังจากนั้น คลิกที่สร้างดราฟใหม่ 
หลังจากนั้นเว็บไซต์จะต้องการทราบวันแรกของการเปิดภาคเรียนเพื่อนำไปสร้างปฏิทินการศึกษา
โดยหน้า Document จะเป็นหน้าที่ใช้จัดการกับร่างปฏิทินทั้งหมดที่ผู้ใช้ได้สร้างไว้
หลังจากที่ผู้ใช้ได้คลิกสร้างปฏิทินขึ้นมาใหม่ ระบบจะต้องการให้ผู้ใช้กรอกข้อมูลของวันเปิดเทอม
ของปีการศึกษานั้น หลังจากนั้นระบบจะทำการสร้างร่างปฏิทินการศึกษาแบบอัตโนมัติ
เพื่อทำให้ง่ายต่อการแก้ไข ไม่เกิดความยุ่งยากในการต้องมาเพิ่มกิจกรรมทีละวันกิจกรรม
โดยกิจกรรมที่นำไปใส่ลงในปฏิทินแบบอัติโนมัตินั้นจะได้มาจากการคำนวน
วันที่อยู่ห่างจากวันเปิดเทอมตามเงื่อนไขของปฏิทิน
% วางรูป Dupilicate flow\
% รูปที่3.2 การทำงานเมื่อผู้ใช้ต้องการทำซ้ำปฏิทินเดิม

จากรูปที่3.2 ผู้ใช้ต้องการจะทำซ้ำ หลังจากที่ผู้ใช้อยู่ในหน้า Document และคลิกที่ทำซ้ำ
เว็บไซต์จะต้องการให้ผู้ใช้กรอกชื่อของปฏิทินที่จะสร้างใหม่ที่ทำซ้ำมาจากปฏิทินเดิม
และปีที่ต้องการเปลี่ยนใหม่ หากมีกิจกรรมของปฏิทินเดิมที่คล้ายคลึงกับปฏิทินของปีถัดไป 
ผู้ใช้สามารถทำซ้ำปฏิทินเดิมแล้วเปลี่ยนเป็นปีถัดไปได้เลย

\subsection{โครงสร้างโปรแกรม}
ในส่วนของ Client จะใช้ภาษา React.JS ในการสร้างเว็บไซต์ แพลตฟอร์มนี้จะใช้กับ
คอมพิวเตอร์ โดยมี Node.JS ในส่วน Backend และใช้ API ในการรับส่งกับฐานข้อมูล
และฐานข้อมูล MongoDB
\chapter{\ifproject%
\ifenglish Experimentation and Results\else การทดลองและผลลัพธ์\fi
\else%
\ifenglish System Evaluation\else การประเมินระบบ\fi
\fi}
\section{ทดสอบการลงชื่อเข้าใช้} 
   การทดสอบเพื่อ ตรวจสอบว่าผู้ที่เข้าใช้งานเว็บไซต์ เป็นบุคลากรของสำนักทะเบียน มหาวิทยาลัยเชียงใหม่ 
และเมื่อผู้เข้าใช้งานต้องการที่จะเข้าสู่ระบบ log in สามารถเข้าถึงเว็บไซต์ได้ตามปกติ

\section{ทดสอบความแม่นยำและประสืทธิภาพการใช้งานของการวางร่างปฏิทินแบบอัตโนมัติ}
   นำปฏิทินการศึกษาของปีที่ผ่านมามาทดสอบ โดยทดสอบให้ผู้ใช้สร้างปฏิทินการศึกษาโดยนำข้อมูลของปฏิทินจากปีที่ผ่านมา 
เพื่อให้ร่างปฏิทินการศึกษาโดยมีการกำหนดวันกิจกรรมแบบอัตโนมัติ แล้วจึงนำมาเปรียบเทียบกับปฏิทินการศึกษาที่เคยร่างไว้แล้ว 
เมื่อเปรียบเทียบกันแล้วผลลัพธ์ของการเปรียบเทียบที่ได้ปฏิทินการศึกษาที่จัดการโดยอัติโนมัติจะต้องไม่ด้อยไปกว่าปฏิทินการศึกษาที่เคยร่างไว้
โดยผู้ที่เป็นคนตัดสินใจว่าไม่ด้อยกว่านั้นเป็นผู้ที่จะมาใช้งานเว็บไซต์ หรือก็คือบุคลากรของสำนักทะเบียน

\section{การประเมินผลระบบ UX/UI} 


\CIreply{UX evaluation?}
\CIreply{performance evaluation?}

\ifproject
\include{chapters/conclusion}
\fi

\CIreply{use BibTeX entries properly!}
% \bibliography{sampleReport}
\begin{thebibliography}{1}
   
  \bibitem{ Jirayut Intachai}
  Jirayut Intachai. (2565). HTML คืออะไร? ทำไมคนเขียนเว็บไซต์ต้องใช้งาน. สืบค้น 16 กุมภาพันธ์ 2565,
  จาก https://goterrestrial.com/2021/05/19/what-is-html/

  \bibitem{ Kongvut Sangkla}
  Kongvut Sangkla. (2565). ทำไมควรเรียนรู้ภาษา JavaScript ?. สืบค้น 16 กุมภาพันธ์ 2565,
  จาก https://blog.2my.xyz/2020/03/10/javascript-from-zero-to-hero/

  \bibitem{ Meta Platforms}
  Meta Platforms. (2565).  React. สืบค้น 16 กุมภาพันธ์ 2565,
  จาก https://reactjs.org/docs/getting-started.html
  
  \bibitem{ Narator Game}
  Narator Game. (2565). MongoDB คืออะไร?. สืบค้น 16 กุมภาพันธ์ 2565,
  จาก https://medium.com/@narator.game/mongodb-คืออะไร-5a0844d7a3c8
  
  \bibitem{ OpenJS Foundation}
  OpenJS Foundation. (2565).  About Node.js®. สืบค้น 16 กุมภาพันธ์ 2565,
  จาก https://nodejs.org/en/about/
  
  \bibitem{Mitsumasa}
  Mitsumasa. (2565). JSON คืออะไร. สืบค้น 16 กุมภาพันธ์ 2565,
  จาก http://www.tutorialdev.com/download/json-%E0%B8%84%E0%B8%B7%E0%B8%AD%E0%B8%AD%E0%B8%B0%E0%B9%84%E0%B8%A3/

\end{thebibliography}

\ifproject
\appendix
\include{chapters/appendix}

%% Display glossary (optional) -- need glossary option.
\ifglossary\glossarypage\fi

%% Display index (optional) -- need idx option.
\ifindex\indexpage\fi

\begin{biosketch}
\begin{center}
  \includegraphics[width=1.5in]{mugshot.jpg}
\end{center}
Your biosketch goes here. Make sure it sits inside
the \texttt{biosketch} environment.
\end{biosketch}
\fi % \ifproject
\end{document}
