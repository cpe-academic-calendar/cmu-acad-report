\chapter{\ifenglish Introduction\else บทนำ\fi}

\section{\ifenglish Project rationale\else ที่มาของโครงงาน\fi}
ในแต่ละปีการศึกษา สำนักทะเบียนและประมวลผล มหาวิทยาลัยเชียงใหม่ จำเป็นจะต้องจัดทำร่างปฏิทินการศึกษาสำหรับปีการศึกษาถัดไป เพื่อให้กรรมการบริหารมหาวิทยาลัยอนุมัติล่วงหน้า

ปฏิทินการศึกษาประกอบไปด้วยกำหนดการของกิจกรรมการศึกษาต่างๆ เช่น วันเปิดภาคเรียน วันลงทะเบียนเรียน วันสุดท้ายของการถอนกระบวนวิชา และวันสอบ เป็นต้น
%
กิจกรรมการศึกษาต่างๆ ส่วนใหญ่นั้นจะถูกกำหนดเป็นเงื่อนไขที่อ้างอิงกับวันเปิดภาคเรียน เช่น วันสุดท้ายของการเรียนการสอน จะถัดจากวันเปิดภาคเรียนประมาณ 16 สัปดาห์ จากนั้น จะเป็นการสอบปลายภาค ระยะเวลา 2 สัปดาห์ แล้วตามด้วยวันประกาศผลการศึกษา หลังจากสอบปลายภาควันสุดท้ายไปประมาณ 2 สัปดาห์
%
จะเห็นว่า หากกำหนดวันเปิดภาคการศึกษาให้ชัดเจนแล้ว กิจกรรมอื่นๆ จะสามารถจัดวางได้โดยอัตโนมัติ จึงทำให้การร่างปฏิทินการศึกษานั้นไม่ควรใช้เวลามากนัก

แต่ในความเป็นจริงแล้ว สำนักทะเบียนและประมวลผลยังขาดเครื่องมือที่จะอำนวยความสะดวกในการร่างปฏิทินการศึกษา ทำให้ต้องใช้เวลาในการสร้างแต่ละร่างถึง 2 สัปดาห์เป็นอย่างต่ำ
%
สาเหตุหลักๆ ในความล่าช้าดังกล่าว คือเงื่อนไขสำหรับกิจกรรมการศึกษาต่างๆ ที่ไม่ได้ระบุไว้เป็นลายลักษณ์อักษรให้ชัดเจน เพื่อที่จะสามารถนำมาใช้ซ้ำได้ ทำให้ผู้จัดทำร่างปฏิทินต้องกำหนดเงื่อนไขดังกล่าวในทุกๆ ปี ก่อนจะวางโครงร่างปฏิทินโดยการนับวันด้วยมือ
%
นอกจากนี้ หากกรรมการบริหารมหาวิทยาลัยมีมติให้แก้ไขร่างดังกล่าว ซึ่งอาจจะเกิดขึ้นได้หลายครั้งในแต่ละปีการศึกษา จะทำให้ผู้จัดทำร่างปฏิทินเสียเวลาเพิ่มเติมมากกว่าที่ควรจะเป็น เนื่องจากจะต้องเริ่มกระบวนการร่างปฏิทินใหม่ทั้งหมด
%
รวมไปปัญหาการที่จะแก้ไขวันหรือกิจกรรมในปฏิทินอย่างกระทัน ก็ยังไม่มีเครื่องมือที่จะช่วยแก้ปัญหาได้ หากกรรมการบริหารมหาวิทยาลัยเชียงใหม่ต้องการที่จะเปรียบเทียบหรือดูตัวอย่างของปฏิทินปีก่อนๆ แล้วนำมาแก้ไขเพียงบางส่วน ผู้จัดทำร่างปฏิทินก็จะต้องเตรียมเอกสารปฏิทินมาหลายฉบับ ซึ่งทำให้เกิดปัญหาให้แก่ผู้จัดทำร่างปฏิทินเป็นอย่างมาก

จากปัญหาการสร้างปฏิทินการศึกษาข้างต้น ผู้จัดทำโครงงานจึงมีแนวคิดที่จะสร้างโปรแกรมวางแผนปฏิทินการศึกษา มหาวิทยาลัยเชียงใหม่  
%
เพื่อที่จะเป็นเครื่องมือที่จะช่วยให้ผู้จัดทำร่างปฏิทินของสำนักทะเบียนสามารถลดเวลาในการจัดทำปฏิทินลงได้
%
อีกทั้งยังสามารถเพิ่มความสะดวกในการจัดทำร่างปฏิทิน ไม่ว่าจะเป็นการเปลี่ยนแปลงวันกิจกรรมได้อย่างรวดเร็ว การแก้ไขร่างปฏิทินอย่างกระทันหัน การสร้างร่างปฏิทิน\\
%
หลายฉบับที่มีความคล้ายคลึงกัน โปรแกรมดังกล่าวก็สามารถตอบสนองได้ 

%     ในปัจจุบันปฏิทินการศึกษาของมหาวิทยาลัยเชียงใหม่นั้นจัดทำโดยทางสำนักทะเบียนของมหาวิทยาลัยเชียงใหม่ 
% โดยที่ในการที่จะสร้างปฏิทินขึ้นมาได้นั้น จะต้องมีการร่างโครงร่างของปฏิทินออกมา โดยที่ในการที่จะร่างปฏิทินนั้ก็จะต้องมีเงื่อนไขในการส้รางปฏิทินต่างๆ ไม่ว่าจะเป็น การกำหนดวันเปิดภาคเรียน การกำหนดระยะเวลาของคาบเรียน วันปิดภาคเรียน ไปจนถึงวันลงทะเบียน 
% เราได้พบปัญหาว่าใน 1 ปฏิทินการศึกษานั้น ใช้เวลาในการร่างปฏิทินไม่ต่ำกว่า 2 สัปดาห์ เนื่องจากทางผู้จัดทำต้องกำหนดเงื่อนไขที่ได้กล่าวข้างต้น ถึงจะทำการวางโครงร่างของปฏิทินได้
% และเมื่อสามารถวางโครงร่างได้แล้วก็จะต้องนำโครงร่างไปส่งให้เป็นที่พิจารณาแก่คณะกรรมการบริหารของมหาวิทยาลัยเชียงใหม่ ซึ่งจะทำให้ต้องมีการแก้ไขอยู่หลายๆครั้ง จึงทำให้การทำปฏิทินวันเปิดภาคเรียนนั้นใช้เวลานานมากเกินไป เราจึงได้เล็งปัญหาของการสร้างปฏทินการศึกษา
% ซึ่งจัดทำโดยสำนักทะเบียนมหาวิทยาลัยเชียงใหม่ จึงได้เกิดเป็นโปรแกรมวางแผนปฏิทินการศึกษา มหาวิทยาลัยเชียงใหม่


\section{\ifenglish Objectives\else วัตถุประสงค์ของโครงงาน\fi}
\begin{enumerate}
    \item เพื่อลดเวลาในการจัดทำปฏิทินการศึกษา
    \item เพื่อสร้างระบบที่สามารถระบุเงื่อนไขต่างๆ ที่จำเป็นต่อการสร้างปฏิทินการศึกษาและสามารถแก้ไขได้ตามความต้องการ
    \item เพื่อสร้างระบบที่สามารถคัดลอกและทำซ้ำของปฏิทินได้ เมื่อต้องการที่จะแก้ปฏิทินหลายๆ ฉบับ และต้องเปลี่ยนการเปลี่ยนแปลงเพียงบางส่วน
\end{enumerate}

\section{\ifenglish Project scope\else ขอบเขตของโครงงาน\fi}

\subsection{\ifenglish Hardware scope\else ขอบเขตด้านฮาร์ดแวร์\fi}
\begin{enumerate}
\item โปรแกรมวางแผนปฏิทินการศึกษานี้สามารถใช้งานได้กับทุกอุปกรณ์ที่เข้าถึง web browser ได้ 
\end{enumerate}

\subsection{\ifenglish Software scope\else ขอบเขตด้านซอฟต์แวร์\fi}
\begin{enumerate}
\item โปรแกรมวางแผนเป็นโปรแกรมนี้จะเพิ่มวันหยุดและกิจกรรมมาให้โดยอัตโนมัติ แต่กิจกรรมที่นักศึกษาเป็นฝ่ายจัดจะไม่นับลงไปด้วย เช่น กิจกรรม Sports Day กิจกรรม Freshy Night เป็นต้น
\item ในการนำออกไฟล์ของโปรแกรมปฏิทิทินการศึกษานี้จะนำออกไฟล์มาเป็นไฟล์ .pdf และ ไฟล์สกุลของโปรแกรม Excel 
\item โปรแกรมวางแผนปฏิทินการศึกษานี้สามารถเข้าถึงได้เฉพาะบุคลากรของสำนักทะเบียน \\ มหาวิทยาลัยเชียงใหม่ที่มีชื่ออยู่ในระบบของ CMU Account เท่านั้น   
\end{enumerate}

\section{\ifenglish Expected outcomes\else ประโยชน์ที่ได้รับ\fi}
\begin{enumerate}
\item สามารถลดเวลาในการร่างปฏิทินการศึกษาให้ใช้เวลาในการทำลดลง
\item สามารถแก้ไขปฏิทินในที่ประชุมได้สะดวกขึ้นหากต้องการแก้กระทันหัน
\item โปรแกรมวางแผนปฏิทินการศึกษานี้สามารถใช้ได้จริง และเป็นประโยชน์ในการออกปฏิทินของสำนักทะเบียนและประมวลผล มหาวิทยาลัยเชียงใหม่
\end{enumerate}

\section{\ifenglish Technology and tools\else เทคโนโลยีและเครื่องมือที่ใช้\fi}

\subsection{\ifenglish Software technology\else เทคโนโลยีด้านซอฟต์แวร์\fi}
\begin{enumerate}
\item ใช้ Figma ในการออกแบบ
\item HTML เป็นภาษาที่ใข้ในการเขียนเว็บ 
\item ในส่วนของ front-end ใช้ React Js 
\item NodeJs ใช้ในการสร้าง web application ในส่วน back-end 
\item MongoDB เป็นเครื่องมือที่ใช้ในการจัดเก็บฐานข้อมูล
\end{enumerate}



\pagebreak
\section{\ifenglish Project plan\else แผนการดำเนินงาน\fi}
\begin{plan}{1}{2022}{4}{2023}
    \planitem{1}{2022}{2}{2022}{ออกแบบ user interface และ user experience}
    \planitem{2}{2022}{3}{2022}{ออกแบบระบบฐานข้อมูล}
    \planitem{4}{2022}{1}{2023}{พัฒนาระบบ front-end และ back-end}
    \planitem{1}{2023}{3}{2023}{ทดลองระบบ}
    \planitem{4}{2023}{4}{2023}{นำเสนอและสรุปผลของการพัฒนาโปรแกรม}
\end{plan}

\section{\ifenglish Roles and responsibilities\else บทบาทและความรับผิดชอบ\fi}
บทบาทในส่วนของ web application มีการแบ่งออกเป็นสองฝั่ง ได้แก่ 
%
ฝั่งของหน้าบ้าน (front-end) ซึ่งเป็นฝั่งที่จำเป็นจะต้องรู้ในเรื่องของ HTML, CSS, JS มีความใจในส่วนของ UX/UI เพื่อออกแบบให้ผู้ใช้งานสามารถเข้าใจได้ง่าย 
%
รวมไปถึงการส่ง requests ส่งไปฝั่ง back-end ผ่าน API ที่จะต้องมีการรับ requests จาก front-end เช่น การดึงข้อมูลมาแสดงผล (GET) การแก้ไขข้อมูลบนฐานข้อมูล (PUT)  
ในส่วนของฝั่งหลังบ้าน (back-end) จำเป็นต้องจัดการในส่วนของฐานข้อมูลด้วย
%


ในช่วงแรกของการทำส่วนของ front-end UX/UI design นายเอื้อบุญ เรือนคำฟู เป็นผู้รับผิดชอบ  
%
ส่วนในฝั่งของ back-end นายเจษฎา จินะกะ เป็นผู้รับผิดชอบ

\section{\ifenglish%
Impacts of this project on society, health, safety, legal, and cultural issues
\else%
ผลกระทบด้านสังคม สุขภาพ ความปลอดภัย กฎหมาย และวัฒนธรรม
\fi}

\subsection{นักศึกษา}
    จากจุดประสงค์ของโครงงาน ผู้พัฒนาต้องการลดเวลาในการสร้างร่างปฏิทินการศึกษา โดยนักศึกษาจะได้รับผลกระทบ ซึ่งผลกระทบ
%
นั้นเกิดจากการลดเวลาในการร่างปฏิทินการศึกษา หากมหาวิทยาลัยมีการประกาศกำหนดการที่เร็วขึ้น นักศึกษาจะสามารถทราบกำหนดการและสามารถจัดเวลาเรียนของตัวเองได้อย่างเหมาะสม 
%

\subsection{มหาวิทยาลัยเชียงใหม่}
สืบเนื่องมาจากการทำให้การร่างปฏิทินการศึกษาง่ายต่อการแก้ไข ทำให้การประกาศการเปลี่ยนแปลงวันหยุดราชการต่างๆ หรือกำหนดการต่างๆ 
%
(ตัวอย่างเช่น กำหนดการณ์สอบ O-Net หรือการสอบต่างๆ ที่สำคัญ ที่สามารถเปลี่ยนแปลงได้ตลอดเวลา การสอบข้างต้นมีผลต่อการเปิดเทอมของมหาวิทยาลัย เนื่องจากเป็น\\
%
เกณฑ์ในการรับนักศึกษาชั้นปีที่ 1 เข้ามาศึกษาในมหาวิทยาลัย) 
%
ที่เป็นเรื่องที่ต้องทำการแก้ไขในกำหนดการของร่างปฏิทิน จะสามารถนำมาแก้ไขในปฏิทินการศึกษาที่ทำการร่างไว้ได้สะดวกและประหยัดเวลาได้มากขึ้น

\subsection{สำนักทะเบียนและประมวลผล มหาวิทยาลัยเชียงใหม่}
จากที่ได้ติดต่อกับบุคลากรจากสำนักทะเบียนและประมวลผล มหาวิทยาลัยเชียงใหม่ พบว่าในการเริ่มทำปฏิทินเกิดจากการทำมือ และทำจากโปรแกรมคอมพิวเตอร์ต่างๆ ที่เป็นโปรแกรมจัดทำเอกสาร
%
หรือโปรแกรมสำหรับเอกสาร มาร่างกำหนดการของปฏิทินการศึกษาและใช้เวลาทำนานนับเดือน นอกจากนั้น หากนำไปเสนอในที่ประชุมจะต้องทำการแก้ไขในหลายๆ ร่าง จึงทำให้เสียเวลาในการทำส่วนอื่น
%
โครงงานนี้จึงจัดทำมาเพื่อสำนักทะเบียนและประมวลผล มหาวิทยาลัยเชียงใหม่ ซึ่งทำให้สะดวกต่อการแก้ไขร่างมากยิ่งขึ้น และประหยัดเวลาในการร่างปฏิทินอีกด้วย โดยฟังก์ชันการวางร่างปฏิทินอัตโนมัติ ทำให้ผู้จัดทำไม่ต้องมานั่งวางทีละวัน แต่จะเป็นการวางวันโดยอัตโนมัติแทน และง่ายต่อการทำซ้ำหรือแก้ไขหลายๆ ร่าง
