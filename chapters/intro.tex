\chapter{\ifenglish Introduction\else บทนำ\fi}

\section{\ifenglish Project rationale\else ที่มาของโครงงาน\fi}
ในแต่ละปีการศึกษา สำนักทะเบียนและประมวลผล มหาวิทยาลัยเชียงใหม่ จำเป็นจะต้องจัดทำร่างปฏิทินการศึกษาสำหรับปีการศึกษาถัดไป เพื่อให้กรรมการบริหารมหาวิทยาลัยอนุมัติล่วงหน้า \\
%
ปฏิทินการศึกษาประกอบไปด้วยกำหนดการของกิจกรรมการศึกษาต่างๆ เช่น วันเปิดภาคเรียน วันลงทะเบียนเรียน วันสุดท้ายของการถอนกระบวนวิชา และวันสอบ เป็นต้น
%
กิจกรรมการศึกษาต่างๆ ส่วนใหญ่นั้นจะถูกกำหนดเป็นเงื่อนไขที่อ้างอิงกับวันเปิดภาคเรียน เช่น วันสุดท้ายของการเรียนการสอน จะถัดจากวันเปิดภาคเรียนประมาณ 16 สัปดาห์ จากนั้น จะเป็นการสอบปลายภาค ระยะเวลา 2 สัปดาห์ แล้วตามด้วยวันประกาศผลการศึกษา หลังจากสอบปลายภาควันสุดท้ายไปประมาณ 2 สัปดาห์
%
จะเห็นว่า หากกำหนดวันเปิดภาคการศึกษาให้ชัดเจนแล้ว กิจกรรมอื่นๆ จะสามารถจัดวางได้โดยอัตโนมัติ จึงทำให้การร่างปฏิทินการศึกษานั้นไม่ควรใช้เวลามากนัก

แต่ในความเป็นจริงแล้ว สำนักทะเบียนและประมวลผลยังขาดเครื่องมือที่จะอำนวยความสะดวกในการร่างปฏิทินการศึกษา ทำให้ต้องใช้เวลาในการสร้างแต่ละร่างถึง 2 สัปดาห์เป็นอย่างต่ำ
%
สาเหตุหลักๆ ในความล่าช้าดังกล่าว คือเงื่อนไขสำหรับกิจกรรมการศึกษาต่างๆ ที่ไม่ได้ระบุไว้เป็นลายลักษณ์อักษรให้ชัดเจน เพื่อที่จะสามารถนำมาใช้ซ้ำได้ ทำให้ผู้จัดทำร่างปฏิทินต้องกำหนดเงื่อนไขดังกล่าวในทุกๆ ปี ก่อนจะวางโครงร่างปฏิทินโดยการนับวันด้วยมือ
%
นอกจากนี้ หากกรรมการบริหารมหาวิทยาลัยมีมติให้แก้ไขร่างดังกล่าว ซึ่งอาจจะเกิดขึ้นได้หลายครั้งในแต่ละปีการศึกษา จะทำให้ผู้จัดทำร่างปฏิทินเสียเวลาเพิ่มเติมมากกว่าที่ควรจะเป็น เนื่องจากจะต้องเริ่มกระบวนการร่างปฏิทินใหม่ทั้งหมด
%\CIreply{มีปัญหาอื่นๆ ที่เกี่ยวข้องอีกหรือไม่}

%จากปัญหาการสร้างปฏิทินการศึกษาข้างต้น ผู้จัดทำโครงงานจึงมีแนวคิดที่จะสร้างโปรแกรมวางแผนปฏิทินการศึกษา มหาวิทยาลัยเชียงใหม่ \CIreply{เพื่อ?}

%     ในปัจจุบันปฏิทินการศึกษาของมหาวิทยาลัยเชียงใหม่นั้นจัดทำโดยทางสำนักทะเบียนของมหาวิทยาลัยเชียงใหม่ 
% โดยที่ในการที่จะสร้างปฏิทินขึ้นมาได้นั้น จะต้องมีการร่างโครงร่างของปฏิทินออกมา โดยที่ในการที่จะร่างปฏิทินนั้ก็จะต้องมีเงื่อนไขในการส้รางปฏิทินต่างๆ ไม่ว่าจะเป็น การกำหนดวันเปิดภาคเรียน การกำหนดระยะเวลาของคาบเรียน วันปิดภาคเรียน ไปจนถึงวันลงทะเบียน 
% เราได้พบปัญหาว่าใน 1 ปฏิทินการศึกษานั้น ใช้เวลาในการร่างปฏิทินไม่ต่ำกว่า 2 สัปดาห์ เนื่องจากทางผู้จัดทำต้องกำหนดเงื่อนไขที่ได้กล่าวข้างต้น ถึงจะทำการวางโครงร่างของปฏิทินได้
% และเมื่อสามารถวางโครงร่างได้แล้วก็จะต้องนำโครงร่างไปส่งให้เป็นที่พิจารณาแก่คณะกรรมการบริหารของมหาวิทยาลัยเชียงใหม่ ซึ่งจะทำให้ต้องมีการแก้ไขอยู่หลายๆครั้ง จึงทำให้การทำปฏิทินวันเปิดภาคเรียนนั้นใช้เวลานานมากเกินไป เราจึงได้เล็งปัญหาของการสร้างปฏทินการศึกษา
% ซึ่งจัดทำโดยสำนักทะเบียนมหาวิทยาลัยเชียงใหม่ จึงได้เกิดเป็นโปรแกรมวางแผนปฏิทินการศึกษา มหาวิทยาลัยเชียงใหม่


\section{\ifenglish Objectives\else วัตถุประสงค์ของโครงงาน\fi}
\begin{enumerate}
    \item เพื่อลดเวลาในการจัดทำปฏิทินการศึกษา
    \item เพื่อสร้างระบบที่สามารถระบุเงื่อนไขต่างๆ ที่จำเป็นต่อการสร้างปฏิทินการศึกษาและสามารถแก้ไขได้ตามความต้องการ
%    \item เพื่อสร้างระบบที่สามารถคัดลอกและทำซ้ำของปฏิทินได้ เมื่อต้องการที่จะแก้ปฏิทินหลายๆ ฉบับ และต้องเปลี่ยนการเปลี่ยนแปลงเพียงบางส่วน\CIreply{ยังไม่เคยพูดถึงในที่มา}
\end{enumerate}

\section{\ifenglish Project scope\else ขอบเขตของโครงงาน\fi}

\subsection{\ifenglish Hardware scope\else ขอบเขตด้านฮาร์ดแวร์\fi}
\begin{enumerate}
\item โปรแกรมวางแผนปฏิทินการศึกษานี้สามารถใช้งานได้กับทุกอุปกรณ์ที่เข้าถึง web browser ได้ 
\end{enumerate}

\subsection{\ifenglish Software scope\else ขอบเขตด้านซอฟต์แวร์\fi}
\begin{enumerate}
\item โปรแกรมวางแผนเป็นโปรแกรมนี้จะเพิ่มวันหยุดและกิจกรรมมาให้โดยอัตโนมัติ แต่กิจกรรมที่นักศึกษาเป็นฝ่ายจัดจะไม่นับลงไปด้วย เช่น กิจกรรม Sports Day กิจกรรม Freshy Night เป็นต้น
\item ในการนำออกไฟล์ของโปรแกรมปฏิทิทินการศึกษานี้จะนำออกไฟล์มาเป็นไฟล์ .pdf, .ics และ ไฟล์สกุล 
\item โปรแกรมวางแผนปฏิทินการศึกษานี้สามารถเข้าถึงได้เฉพาะบุคลากรของสำนักทะเบียน \\ มหาวิทยาลัยเชียงใหม่ที่มีชื่ออยู่ในระบบของ CMU Account เท่านั้น   
\end{enumerate}

\section{\ifenglish Expected outcomes\else ประโยชน์ที่ได้รับ\fi}
\begin{enumerate}
\item สามารถลดเวลาในการร่างปฏิทินการศึกษาให้ใช้เวลาในการทำลดลง
%\item \CI{สามารถแก้ไขปฏิทินในที่ประชุมได้สะดวกขึ้นหากต้องการแก้กระทันหัน}{ยังไม่เคยพูดถึงประเด็นนี้}
\item โปรแกรมวางแผนปฏิทินการศึกษานี้สามารถใช้ได้จริงและเป็นประโยชน์ในการออกปฏิทินของ \\ สำนักทะเบียนและประมวลผล มหาวิทยาลัยเชียงใหม่
\end{enumerate}

\section{\ifenglish Technology and tools\else เทคโนโลยีและเครื่องมือที่ใช้\fi}


\subsection{\ifenglish Software technology\else เทคโนโลยีด้านซอฟต์แวร์\fi}
\begin{enumerate}
\item ใช้ Figma ในการออกแบบ
\item HTML เป็นภาษาที่ใข้ในการเขียนเว็บ 
\item ในส่วนของ front-end ใช้ React Js 
\item NodeJs ใช้ในการสร้าง web application ในส่วน back-end 
\item MongoDB เป็นเครื่องมือที่ใช้ในการจัดเก็บฐานข้อมูล
\end{enumerate}

\section{\ifenglish Project plan\else แผนการดำเนินงาน\fi}

\begin{plan}{1}{2022}{4}{2023}
    \planitem{1}{2022}{2}{2022}{ออกแบบ user interface และ user experience}
    \planitem{2}{2022}{3}{2022}{ออกแบบระบบฐานข้อมูล}
    \planitem{4}{2022}{1}{2023}{พัฒนาระบบ front-end และ back-end}
    \planitem{1}{2023}{3}{2023}{ทดลองระบบ}
    \planitem{4}{2023}{4}{2023}{นำเสนอและสรุปผลของการพัฒนาโปรแกรม}
\end{plan}

\section{\ifenglish Roles and responsibilities\else บทบาทและความรับผิดชอบ\fi}
    บทบาทในส่วนของ web application มีการแบ่งออกเป็นสองฝั่ง ได้แก่ 
%
ฝั่งของหน้าบ้าน(front-end) ซึ่งเป็นฝั่งที่จำเป็นจะต้องรู้ในเรื่องของ HTML,CSS, JS มีความใจในส่วนของ UX/UI เพื่อออกแบบให้ผู้ใช้งานสามารถเจ้าใจได้ง่าย 
%
รวมไปถึงการส่ง requests ส่งไปฝั่ง back-end โดยในส่วนของฝั่งหลังบ้าน(back-end) จำเป็นต้องจัดการในส่วนของฐานข้อมูล และมีความรู้ในเรื่องของการเขียน API 
%
โดยงานส่วนใหญ่จะหนักไปทางฝั่งของ front-end โดยนายเจษฎา จินะกะ และ นายเอื้อบุญ เรือนคำฟู รับผิดชอบร่วมกัน 

\section{\ifenglish%
Impacts of this project on society, health, safety, legal, and cultural issues
\else%
ผลกระทบด้านสังคม สุขภาพ ความปลอดภัย กฎหมาย และวัฒนธรรม
\fi}

แนวทางและโยชน์ในการประยุกต์ใช้งานโครงงานกับงานในด้านอื่นๆ รวมถึงผลกระทบในด้านสังคมและสิ่งแวดล้อมจากการใช้ความรู้ทางวิศวกรรมที่ได้
