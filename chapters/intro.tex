\chapter{\ifenglish Introduction\else บทนำ\fi}

\section{\ifenglish Project rationale\else ที่มาของโครงงาน\fi}
ในปัจจุบันปฏิทินการศึกษาของมหาวิทยาลัยเชียงใหม่นั้นจัดทำโดยทางสำนักทะเบียนของมหาวิทยาลัยเชียงใหม่ 
โดยที่ในการที่จะสร้างปฏิทินขึ้นมาได้นั้น จะต้องมีการร่างโครงร่างของปฏิทินออกมา โดยที่ในการที่จะร่างปฏิทินนั้ก็จะต้องมีเงื่อนไขในการส้รางปฏิทินต่างๆ ไม่ว่าจะเป็น การกำหนดวันเปิดภาคเรียน การกำหนดระยะเวลาของคาบเรียน วันปิดภาคเรียน ไปจนถึงวันลงทะเบียน 
เราได้พบปัญหาว่าใน 1 ปฏิทินการศึกษานั้น ใช้เวลาในการร่างปฏิทินไม่ต่ำกว่า 2 สัปดาห์ 

เนื่องจากทางผู้จัดทำต้องกำหนดเงื่อนไขที่ได้กล่าวข้างต้น ถึงจะทำการวางโครงร่างของปฏิทินได้
และเมื่อสามารถวางโครงร่างได้แล้วก็จะต้องนำโครงร่างไปส่งให้เป็นที่พิจารณาแก่คณะกรรมการบริหารของมหาวิทยาลัยเชียงใหม่ ซึ่งจะทำให้ต้องมีการแก้ไขอยู่หลายๆครั้ง จึงทำให้การทำปฏิทินวันเปิดภาคเรียนนั้นใช้เวลานานมากเกินไป เราจึงได้เล็งปัญหาของการสร้างปฏทินการศึกษา
ซึ่งจัดทำโดยสำนักทะเบียนมหาวิทยาลัยเชียงใหม่ จึงได้เกิดเป็นโปรแกรมวางแผนปฏิทินการศึกษา มหาวิทยาลัยเชียงใหม่


\section{\ifenglish Objectives\else วัตถุประสงค์ของโครงงาน\fi}
\begin{enumerate}
    \item ต้องการลดเวลาในการจัดทำปฏิทินการศึกษาให้น้อยลง
    \item ต้องการระบบที่สามารถระบุเงื่อนไขต่างๆ ที่จำเป็นต่อการสร้างปฏิทินการศึกษาและสามารถแก้ไขได้ตามต้องการ
    \item ต้องการสร้างระบบที่สามารถคัดลอกและทำซ้ำของปฏิทินได้ เมื่อต้องการที่จะแก้ปฏิทินหลายๆฉบับ และต้องเปลี่ยนการเปลี่ยนแปลงเพียงบางส่วน 
\end{enumerate}

\section{\ifenglish Project scope\else ขอบเขตของโครงงาน\fi}

\subsection{\ifenglish Software scope\else ขอบเขตด้านซอฟต์แวร์\fi}
\begin{enumerate}
\item โปรแกรมวางแผนเป็นโปรแกรมนี้จะเพิ่มวันหยุดและกิจกรรมมาให้โดยอัตโนมัติ แต่กิจกรรมที่นักศึกษาเป็นคนจัดจะไม่นับลงไปด้วย ไม่ว่าจะเป็น กิจกรรม Sport Day กิจกรรม Freshy Night เป็นต้น
\item ในการนำออกไฟล์ของโปรแกรมปฏิทิทินการศึกษานี้จะนำออกไฟล์มาเป็นไฟล์ .pdf, .ics และ img 
\item โปรแกรมวางแผนปฏิทินการศึกษานี้สามารถเข้าถึงได้เฉพาะบุคลากรของสำนักทะเบียนมหาวิทยาลัยเชียงใหม่ที่มีชื่อระบบอยู่ในระบบ OAuth เท่านั้น   
\end{enumerate}

\section{\ifenglish Expected outcomes\else ประโยชน์ที่ได้รับ\fi}
\begin{enumerate}
\item สามารถลดเวลาในการร่างปฏิทินการศึกษาให้ใช้เวลาในการทำลดลง
\item สามารถแก้ไขปฏิทินในที่ประชุมได้สะดวกขึ้นหากต้องการแก้กระทันหัน
\item โปรแกรมวางแผนปฏิทินการศึกษานี้สามารถใช้ได้จริงและเป็นประโยชน์ในการออกปฏิทินของสำนักทะเบียนมหาวิทยาลัยเชียงใหม่
\end{enumerate}
\section{\ifenglish Technology and tools\else เทคโนโลยีและเครื่องมือที่ใช้\fi}


\subsection{\ifenglish Software technology\else เทคโนโลยีด้านซอฟต์แวร์\fi}


\section{\ifenglish Project plan\else แผนการดำเนินงาน\fi}

\begin{plan}{1}{2022}{4}{2023}
    \planitem{1}{2022}{2}{2022}{ออกแบบ user interface และ user experience}
    \planitem{2}{2022}{3}{2022}{ออกแบบในส่วนของฐานข้อมูล}
    \planitem{4}{2022}{1}{2023}{พัฒนาระบบ Front-end และ Back-end}
    \planitem{1}{2023}{3}{2023}{ทดลองระบบ}
    \planitem{4}{2023}{4}{2023}{นำเสนอและสรุปผลของการพัฒนาโปรแกรม}
\end{plan}

\section{\ifenglish Roles and responsibilities\else บทบาทและความรับผิดชอบ\fi}
อธิบายว่าในการทำงาน นศ. มีการกำหนดบทบาทและแบ่งหน้าที่งานอย่างไรในการทำงาน จำเป็นต้องใช้ความรู้ใดในการทำงานบ้าง

\section{\ifenglish%
Impacts of this project on society, health, safety, legal, and cultural issues
\else%
ผลกระทบด้านสังคม สุขภาพ ความปลอดภัย กฎหมาย และวัฒนธรรม
\fi}

แนวทางและโยชน์ในการประยุกต์ใช้งานโครงงานกับงานในด้านอื่นๆ รวมถึงผลกระทบในด้านสังคมและสิ่งแวดล้อมจากการใช้ความรู้ทางวิศวกรรมที่ได้
