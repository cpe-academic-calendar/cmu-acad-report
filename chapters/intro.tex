\chapter{\ifenglish Introduction\else บทนำ\fi}

\section{\ifenglish Project rationale\else ที่มาของโครงงาน\fi}
    ในปัจจุบันปฏิทินการศึกษาของมหาวิทยาลัยเชียงใหม่นั้นจัดทำโดยทางสำนักทะเบียนของมหาวิทยาลัยเชียงใหม่ 
โดยที่ในการที่จะสร้างปฏิทินขึ้นมาได้นั้น จะต้องมีการร่างโครงร่างของปฏิทินออกมา โดยที่ในการที่จะร่างปฏิทินนั้ก็จะต้องมีเงื่อนไขในการส้รางปฏิทินต่างๆ ไม่ว่าจะเป็น การกำหนดวันเปิดภาคเรียน การกำหนดระยะเวลาของคาบเรียน วันปิดภาคเรียน ไปจนถึงวันลงทะเบียน 
เราได้พบปัญหาว่าใน 1 ปฏิทินการศึกษานั้น ใช้เวลาในการร่างปฏิทินไม่ต่ำกว่า 2 สัปดาห์ เนื่องจากทางผู้จัดทำต้องกำหนดเงื่อนไขที่ได้กล่าวข้างต้น ถึงจะทำการวางโครงร่างของปฏิทินได้
และเมื่อสามารถวางโครงร่างได้แล้วก็จะต้องนำโครงร่างไปส่งให้เป็นที่พิจารณาแก่คณะกรรมการบริหารของมหาวิทยาลัยเชียงใหม่ ซึ่งจะทำให้ต้องมีการแก้ไขอยู่หลายๆครั้ง จึงทำให้การทำปฏิทินวันเปิดภาคเรียนนั้นใช้เวลานานมากเกินไป เราจึงได้เล็งปัญหาของการสร้างปฏทินการศึกษา
ซึ่งจัดทำโดยสำนักทะเบียนมหาวิทยาลัยเชียงใหม่ จึงได้เกิดเป็นโปรแกรมวางแผนปฏทินการศึกษา มหาวิทยาลัยเชียงใหม่


\section{\ifenglish Objectives\else วัตถุประสงค์ของโครงงาน\fi}
\begin{enumerate}
    \item 
    \item
    \item
    \item
    \item

\end{enumerate}

\section{\ifenglish Project scope\else ขอบเขตของโครงงาน\fi}

\subsection{\ifenglish Hardware scope\else ขอบเขตด้านฮาร์ดแวร์\fi}

\subsection{\ifenglish Software scope\else ขอบเขตด้านซอฟต์แวร์\fi}

\section{\ifenglish Expected outcomes\else ประโยชน์ที่ได้รับ\fi}

\section{\ifenglish Technology and tools\else เทคโนโลยีและเครื่องมือที่ใช้\fi}

\subsection{\ifenglish Hardware technology\else เทคโนโลยีด้านฮาร์ดแวร์\fi}

\subsection{\ifenglish Software technology\else เทคโนโลยีด้านซอฟต์แวร์\fi}

\section{\ifenglish Project plan\else แผนการดำเนินงาน\fi}

\begin{plan}{6}{2020}{2}{2021}
    \planitem{7}{2020}{8}{2020}{ศึกษาค้นคว้า}
    \planitem{8}{2020}{1}{2021}{ชิล}
    \planitem{2}{2021}{2}{2021}{เผา}
    \planitem{12}{2019}{1}{2022}{ทดสอบ}
\end{plan}

\section{\ifenglish Roles and responsibilities\else บทบาทและความรับผิดชอบ\fi}
อธิบายว่าในการทำงาน นศ. มีการกำหนดบทบาทและแบ่งหน้าที่งานอย่างไรในการทำงาน จำเป็นต้องใช้ความรู้ใดในการทำงานบ้าง

\section{\ifenglish%
Impacts of this project on society, health, safety, legal, and cultural issues
\else%
ผลกระทบด้านสังคม สุขภาพ ความปลอดภัย กฎหมาย และวัฒนธรรม
\fi}

แนวทางและโยชน์ในการประยุกต์ใช้งานโครงงานกับงานในด้านอื่นๆ รวมถึงผลกระทบในด้านสังคมและสิ่งแวดล้อมจากการใช้ความรู้ทางวิศวกรรมที่ได้
