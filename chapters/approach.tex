\chapter{\ifproject%
\ifenglish Project Structure and Methodology\else โครงสร้างและขั้นตอนการทำงาน\fi
\else%
\ifenglish Project Structure\else โครงสร้างของโครงงาน\fi
\fi
}

ในบทนี้จะกล่าวถึงหลักการ และการออกแบบระบบไปจนถึงขั้นตอนการออกแบบจากความต้องการของผู้ใช้งาน

\makeatletter

% \renewcommand\section{\@startsection {section}{1}{\z@}%
%                                    {13.5ex \@plus -1ex \@minus -.2ex}%
%                                    {2.3ex \@plus.2ex}%
%                                    {\normalfont\large\bfseries}}

\makeatother
%\vspace{2ex}
% \titleformat{\section}{\normalfont\bfseries}{\thesection}{1em}{}
% \titlespacing*{\section}{0pt}{10ex}{0pt}

\section{การติดต่อและคุยงานเพื่อสรุปความต้องการของสำนักทะเบียน}

\begin{figure}
\begin{center}
% \includegraphics{800px-Briny_Beach.jpg}
\end{center}
% \caption[Poem]{The Walrus and the Carpenter}
\label{fig:walrus}
\end{figure}

% \subsection{The Black Kitten}
  เนื่องจากจุดประสงค์ของโครงงานนี้คือต้องการพัฒนาเว็บไซต์ให้แก่สำนักทะเบียน
  จึงจะต้องเริ่มจากการพูดคุยกับบุคลากรของสำนักทะเบียนเพื่อให้ได้
  ความต้องการที่แก้จริงของโครงงานโดยในปฏิทินจะมีเงื่อนไขต่างๆ อันสรุปได้ดังนี้

  The way Dinah washed her children's faces was this:  first she
held the poor thing down by its ear with one paw, and then with
the other paw she rubbed its face all over, the wrong way,
beginning at the nose:  and just now, as I said, she was hard at
work on the white kitten, which was lying quite still and trying
to purr---no doubt feeling that it was all meant for its good.

หลังจากที่ไดเงื่อนไขทั้งหมดครบแล้ว จะนำเงื่อนไขเหล่านี้มาแยกออกจากระบบที่ตอบสนอง
แบะเพิ่มความสะดวกสบายของผู้ใช้

% \subsection{The Reproach}

%   `Oh, you wicked little thing!' cried Alice, catching up the
% kitten, and giving it a little kiss to make it understand that it
% was in disgrace.  `Really, Dinah ought to have taught you better
% manners!  You OUGHT, Dinah, you know you ought!' she added,
% looking reproachfully at the old cat, and speaking in as cross a
% voice as she could manage---and then she scrambled back into the
% arm-chair, taking the kitten and the worsted with her, and began
% winding up the ball again.  But she didn't get on very fast, as
% she was talking all the time, sometimes to the kitten, and
% sometimes to herself.  Kitty sat very demurely on her knee,
% pretending to watch the progress of the winding, and now and then
% putting out one paw and gently touching the ball, as if it would
% be glad to help, if it might.

%   `Do you know what to-morrow is, Kitty?' Alice began.  `You'd
% have guessed if you'd been up in the window with me---only Dinah
% was making you tidy, so you couldn't.  I was watching the boys
% getting in stick for the bonfire---and it wants plenty of
% sticks, Kitty!  Only it got so cold, and it snowed so, they had
% to leave off.  Never mind, Kitty, we'll go and see the bonfire
% to-morrow.'  Here Alice wound two or three turns of the worsted
% round the kitten's neck, just to see how it would look:  this led
% to a scramble, in which the ball rolled down upon the floor, and
% yards and yards of it got unwound again.

%   `Do you know, I was so angry, Kitty,' Alice went on as soon as
% they were comfortably settled again, `when I saw all the mischief
% you had been doing, I was very nearly opening the window, and
% putting you out into the snow!  And you'd have deserved it, you
% little mischievous darling!  What have you got to say for
% yourself?  Now don't interrupt me!' she went on, holding up one
% finger.  `I'm going to tell you all your faults.  Number one:
% you squeaked twice while Dinah was washing your face this
% morning.  Now you can't deny it, Kitty:  I heard you!  What that
% you say?' (pretending that the kitten was speaking.)  `Her paw
% went into your eye?  Well, that's YOUR fault, for keeping your
% eyes open---if you'd shut them tight up, it wouldn't have
% happened.  Now don't make any more excuses, but listen!  Number
% two:  you pulled Snowdrop away by the tail just as I had put down
% the saucer of milk before her!  What, you were thirsty, were you?

\section{โครงสร้างของโครงงาน และการทำงานของโปแกรม}
\subsection{การทำงานของโปรแกรม}
จากการสรุปความต้องการของสำนักทะเบียนผ่านทางบุคลากรจึงสรุปออกมาเป็น User Flowหรือ
สิ่งที่แสดงเส้นทางของผู้ใช้แอพพลิเคชั่นได้ดังนี้
% \includegraphics{pic3.1.jpg}
% \caption[Poem]{รูปที่3.1 การทำงานเมื่อผู้ใช้ต้องการสร้างแบบปฏิทินใหม่}

จากรูปที่3.1 ผู้ใช้จะเริ่มจากการเข้าสู่ระบบโดยใช้ CMU QAuthหลังจากนั้น คลิกที่สร้างดราฟใหม่ 
หลังจากนั้นเว็บไซต์จะต้องการทราบวันแรกของการเปิดภาคเรียนเพื่อนำไปสร้างปฏิทินการศึกษา
โดยหน้า Document จะเป็นหน้าที่ใช้จัดการกับร่างปฏิทินทั้งหมดที่ผู้ใช้ได้สร้างไว้
หลังจากที่ผู้ใช้ได้คลิกสร้างปฏิทินขึ้นมาใหม่ ระบบจะต้องการให้ผู้ใช้กรอกข้อมูลของวันเปิดเทอม
ของปีการศึกษานั้น หลังจากนั้นระบบจะทำการสร้างร่างปฏิทินการศึกษาแบบอัตโนมัติ
เพื่อทำให้ง่ายต่อการแก้ไข ไม่เกิดความยุ่งยากในการต้องมาเพิ่มกิจกรรมทีละวันกิจกรรม
โดยกิจกรรมที่นำไปใส่ลงในปฏิทินแบบอัติโนมัตินั้นจะได้มาจากการคำนวน
วันที่อยู่ห่างจากวันเปิดเทอมตามเงื่อนไขของปฏิทิน
% วางรูป Dupilicate flow\
% รูปที่3.2 การทำงานเมื่อผู้ใช้ต้องการทำซ้ำปฏิทินเดิม

จากรูปที่3.2 ผู้ใช้ต้องการจะทำซ้ำ หลังจากที่ผู้ใช้อยู่ในหน้า Document และคลิกที่ทำซ้ำ
เว็บไซต์จะต้องการให้ผู้ใช้กรอกชื่อของปฏิทินที่จะสร้างใหม่ที่ทำซ้ำมาจากปฏิทินเดิม
และปีที่ต้องการเปลี่ยนใหม่ หากมีกิจกรรมของปฏิทินเดิมที่คล้ายคลึงกับปฏิทินของปีถัดไป 
ผู้ใช้สามารถทำซ้ำปฏิทินเดิมแล้วเปลี่ยนเป็นปีถัดไปได้เลย

\subsection{โครงสร้างโปรแกรม}
ในส่วนของ Client จะใช้ภาษา React.JS ในการสร้างเว็บไซต์ แพลตฟอร์มนี้จะใช้กับ
คอมพิวเตอร์ โดยมี Node.JS ในส่วน Backend และใช้ API ในการรับส่งกับฐานข้อมูล
และฐานข้อมูล MongoDB