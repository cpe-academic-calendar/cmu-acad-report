\chapter{\ifproject%
\ifenglish Project Structure and Methodology\else โครงสร้างและขั้นตอนการทำงาน\fi
\else%
\ifenglish Project Structure\else โครงสร้างของโครงงาน\fi
\fi
}

ในบทนี้จะกล่าวถึงหลักการ และการออกแบบระบบไปจนถึงขั้นตอนการออกแบบจากความต้องการของผู้ใช้งาน

\makeatletter

% \renewcommand\section{\@startsection {section}{1}{\z@}%
%                                    {13.5ex \@plus -1ex \@minus -.2ex}%
%                                    {2.3ex \@plus.2ex}%
%                                    {\normalfont\large\bfseries}}

\makeatother
%\vspace{2ex}
% \titleformat{\section}{\normalfont\bfseries}{\thesection}{1em}{}
% \titlespacing*{\section}{0pt}{10ex}{0pt}

\section{การติดต่อและคุยงานเพื่อสรุปความต้องการของสำนักทะเบียน}

% \begin{figure}
% \begin{center}
% % \includegraphics{800px-Briny_Beach.jpg}
% \end{center}
% % \caption[Poem]{The Walrus and the Carpenter}
% \label{fig:walrus}
% \end{figure}

\section{เงื่อนไขของการวางร่างปฏิทินการศึกษา}
  อ้างอิงจาก ร่างปฏิทินการศึกษาของสำนักทะเบียนปีการศึกษา 2564{\cite{CMU_Calendar}}
  เนื่องจากจุดประสงค์ของโครงงานนี้คือต้องการพัฒนาเว็บไซต์ให้แก่สำนักทะเบียน
จึงจะต้องเริ่มจากการพูดคุยกับบุคลากรของสำนักทะเบียนเพื่อให้ได้ความต้องการที่แท้จริงของโครงงาน โดยในปฏิทินจะมีเงื่อนไขต่างๆ อันสรุปได้ดังนี้

\subsection{ภาคการศึกษาที่ 1}
 
\begin{itemize}
  \item   คาบเรียนของแต่ละวัน มีจำนวนดังนี้ วันจันทร์อย่างน้อย 14 คาบ วันอังคารอย่างน้อย 14 คาบ วันพุธอย่างน้อย 14 คาบ วันพฤหัสบดี 13 คาบ วันศุกร์ 12 คาบ(โดยประมาณ)  
  \item วันเปิดภาคเรียนมักจะอยู่ในเดือนมิถุนายน
  \item รูปแบบวันจันทร์ พฤหัสบดี มีระยะเวลาการเรียนการสอนตลอดภาคการศึกษา จำนวน 42 ชั่วโมง
  \item เนื่องจากมีจำนวนเวลาตามที่กล่าวไว้ข้างต้น จึงทำให้วันอังคาร ศุกร์ มีระยะเวลาการเรียนการสอนตลอดภาคการศึกษา จำนวน 40 ชั่วโมง 30 นาที 
  \item วันลงทะเบียนเรียนล่วงหน้า สัปดาห์แรกของเดือนก่อนหน้าที่จะเปิดภาคการศึกษา
  \item ประกาศผลการลงทะเบียนเรียนล่วงหน้าหลังจากลงทะเบียนล่วงหน้าประมาณ 10 วัน
  \item วันลงทะเบียนเรียนมักจะเริ่มก่อนวันเปิดภาคเรียนวันแรก 1-2 วัน
  \item วันลงทะเบียนมีระยะเวลา 8 วัน (เป็นเวลาที่ตรงกับวันเรียนด้วย)
  \item วันถอนกระบวนวิชาโดยไม่ได้รับอักษรลำดับขั้น W จะเริ่มหลังจากวันลงทะเบียนเรียน มีระยะเวลา 2 สัปดาห์
  \item วันที่อาจารย์ที่ปรึกษาให้ความเห็นชอบการลงทะเบียนเรียนของนักศึกษาทางระบบ Internet ระยะเวลา 1 สัปดาห์หลังจากเปิดภาคเรียน
  \item หลังจากหมดวันถอนกระบวนวิชาโดยไม่ได้รับอักษรลำดับขั้น W จะเริ่มนับวันถอนกระบวนวิชาโดยได้รับอักษรลำดับขั้น W
  \item หลังจากเปิดภาคเรียน เมื่อเรียนครบ 8 สัปดาห์จะเริ่มสอบกลางภาคในสัปดาห์ที่ 9 โดยมีระยะเวลา 1 สัปดาห์
  \item หลังจากวันสุดท้ายของการสอบกลางภาค เริ่มเรียนครึ่งภาคเรียนหลัง (สัปดาห์ที่ 10) โดยครึ่งภาคเรียนหลัง 7 สัปดาห์
  \item เริ่มสอบปลายภาคในสัปดาห์ที่17และ18(ระยะเวลา 2 สัปดาห์)
  \item หลังสอบปลายภาคจะทำการหยุดเรียนจนจบสัปดาห์นั้นและหยุดเรียนเพิ่มอีก 1 สัปดาห์ และเปิดภาคเรียนที่2 วันจันทร์ของสัปดาห์ถัดมา
  \item วันไห้ครู ไม่มีการเรียนการสอน
  
\end{itemize}

  

\subsection{ภาคการศึกษาที่ 2}
\begin{itemize}
  \item คาบเรียนของแต่ละวัน มีจำนวนดังนี้ วันจันทร์ 12 คาบ วันอังคาร 15 คาบ วันพุธ 15 คาบ วันพฤหัสบดี 15 คาบ วันศุกร์ 14 คาบ
  \item รูปแบบวันจันทร์ พฤหัสบดี มีระยะเวลาการเรียนการสอนตลอดภาคการศึกษา จำนวน 40 ชั่วโมง 30 นาที
  \item รูปแบบวันอังคาร ศุกร์ มีระยะเวลาการเรียนการสอนตลอดภาคการศึกษา จำนวน 43 ชั่วโมง 30 นาที
  \item วันลงทะเบียนเรียนล่วงหน้า หลังจากสอบกลางภาคประมาณ 3 สัปดาห์
  \item ประกาศผลการลงทะเบียนล่วงหน้า ระยะเวลา 1 สัปดาห์ หลังจากวันลงทะเบียนล่วงหน้า 
  \item เปิดภาคการศึกษาที่ 2 (ระยะเวลา 18 สัปดาห์เหมือนภาคการศึกษาที่ 1)
  \item วันลงทะเบียนเรียนมักจะเริ่มก่อนวันเปิดภาคเรียนวันแรก 1--2 วัน
  \item วันลงทะเบียนมีระยะเวลา8วัน (เป็นเวลาที่ตรงกับวันเรียนด้วย)
  \item หลังจากสอบปลายภาคภาคเรียนที่2 จะหยุดเรียน3 สัปดาห์
  \item หลังจากหยุดเรียน3สัปดาห์ จะเริ่มเปิดภาคเรียนฤดูร้อน
\end{itemize}

\subsection{ภาคการศึกษาฤดูร้อน}
\begin{itemize}
  \item ภาคฤดูร้อน มีรูปแบบการเรียนการสอนตั้งแต่วันจันทร์--ศุกร์ ระยะเวลาที่ใช้สำหรับการเรียนการสอนแต่ละกระบวนวิชามีเวลาเรียนจำนวน 29 คาบ เท่ากับ 43 ชั่วโมง 30 นาที
  \item ลงทะเบียนล่วงหน้าของภาคฤดูร้อนเริ่มหลังจากปิดภาคเรียนที่ 2 ไปแล้ว 1 สัปดาห์(ระยะเวลา 4 วัน)
  \item ประกาศผลการลงทะเบียนล่วงหน้าหลังจากปิดการลงทะเบียน 2 วัน
  \item วันเพิ่ม-ถอนกระบวนวิชา/ลงทะเบียน สำหรับนักศึกษาทุกระดับ ก่อนวันเปิดภาคเรียน 1 สัปดาห์
  \item ภาคฤดูร้อนระยะเวลาเรียน 6 สัปดาห์
  \item สัปดาห์ที่ 7 ของภาคฤดูร้อนจะเป็นสอบปลายภาคฤดูร้อน
  \item หลังจากสอบปลายภาคฤดูร้อนแล้วจะปิดปีการศึกษา
\end{itemize}


\section{ฟังก์ชันการสร้างร่างปฏิทินการศึกษาใหม่}
จากการสรุปความต้องการของสำนักทะเบียนผ่านทางบุคลากร จึงสรุปออกมาเป็น user flow หรือสิ่งที่แสดงเส้นทางของผู้ใช้แอพพลิเคชั่นได้ดังรูปที่~\ref{fig:user-flow-new}

%
\begin{figure}[h]
\centering
\includegraphics[width=1\textwidth]{pic3.1.jpg}
\caption{การทำงานเมื่อผู้ใช้ต้องการสร้างแบบปฏิทินใหม่}
\label{fig:user-flow-new}
\end{figure}
%
โดยผู้ใช้จะเริ่มจากการเข้าสู่ระบบโดยใช้ CMU OAuth หลังจากนั้น คลิกที่สร้างดราฟใหม่ 
หลังจากนั้นเว็บไซต์จะต้องการทราบวันแรกของการเปิดภาคเรียนเพื่อนำไปสร้างปฏิทินการศึกษา
โดยหน้า Document จะเป็นหน้าที่ใช้จัดการกับร่างปฏิทินทั้งหมดที่ผู้ใช้ได้สร้างไว้
หลังจากที่ผู้ใช้ได้คลิกสร้างปฏิทินขึ้นมาใหม่ ระบบจะต้องการให้ผู้ใช้กรอกข้อมูลของวันเปิดเทอม
ของปีการศึกษานั้น หลังจากนั้นระบบจะทำการสร้างร่างปฏิทินการศึกษาแบบอัตโนมัติ
เพื่อทำให้ง่ายต่อการแก้ไข ไม่เกิดความยุ่งยากในการต้องมาเพิ่มกิจกรรมทีละวันกิจกรรม
โดยกิจกรรมที่นำไปใส่ลงในปฏิทินแบบอัติโนมัตินั้นจะได้มาจากการคำนวณ
วันที่อยู่ห่างจากวันเปิดเทอมตามเงื่อนไขของปฏิทิน
% วางรูป Dupilicate flow\
% รูปที่3.2 การทำงานเมื่อผู้ใช้ต้องการทำซ้ำปฏิทินเดิม

\section{ฟังก์ชั่นการทำซ้ำปฏิทินการศึกษาใหม่}
\begin{figure}[h]
\centering
\includegraphics[width=1\textwidth]{pic3.2.jpg}
\caption{การทำงานเมื่อผู้ใช้ต้องการทำซ้ำแบบปฏิทินที่เคยทำไว้แล้ว}
\label{fig:user-flow-duplicate}
\end{figure}
จากรูปที่~\ref{fig:user-flow-duplicate} ผู้ใช้ต้องการจะทำซ้ำ หลังจากที่ผู้ใช้อยู่ในหน้า Document และคลิกที่ทำซ้ำหรือ Duplicate
เว็บไซต์จะต้องการให้ผู้ใช้กรอกชื่อของปฏิทินที่จะสร้างใหม่ที่ทำซ้ำมาจากปฏิทินเดิม
และปีที่ต้องการเปลี่ยนใหม่ หากมีกิจกรรมของปฏิทินเดิมที่คล้ายคลึงกับปฏิทินของปีถัดไป 
ผู้ใช้สามารถทำซ้ำปฏิทินเดิมแล้วเปลี่ยนเป็นปีถัดไปได้เลย 
การทำซ้ำด่้วยฟังก์ชันนี้ทำให้ผู้ใช้สามารถสร้างแบบร่างปฏิทินที่มีการวางวันและกิจกรรมเหมือนกับปฏิทินที่เคยทำไว้แล้ว
หรือผู้ใช้จะทำซ้ำปฏิทินเพื่อนำร่างที่ทำซ้ำมาไปแก้ไข แทนที่จะสร้างใหม่และวางวันกิจกรรมแบบเดิม เพื่อทำให้การทำร่างปฏิทินที่คล้ายกันใช้เวลาน้อยลง

\section{ฟังก์ชันการตั้งค่าการใช้งาน}
เนื่องจากวันหยุดราชการตามราชกิจจานุเบกษา และเงื่อนไขของการวางกิจกรรมในปฏิทินการศึกษาสามารถเปลี่ยนแปลงได้ 
จึงทำให้ต้องมีฟังก์ชันการตั้งค่าการใช้งาน เพื่อให้ผู้ใช้สามารถตั้งค่าวันหยุดหรือเงื่อนไขได้หากมีประกาศเพิ่มเติมหรือเปลี่ยนแปลงจากทางส่วนกลางหรือทางมหาวิทยาลัย

\section{โครงสร้างโปรแกรม}
ในส่วนของ client จะใช้ภาษา React.JS ในการสร้างเว็บไซต์ แพลตฟอร์มนี้จะใช้กับ
คอมพิวเตอร์ โดยมี Node.JS ในส่วน backend และใช้ API ในการรับส่งกับฐานข้อมูล
และฐานข้อมูล MongoDB