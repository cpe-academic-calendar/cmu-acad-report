\maketitle
\makesignature

\ifproject
   \begin{abstractTH}
      \par ในแต่ละปีการศึกษา สำนักทะเบียนและประมวลผลมหาวิทยาลัยเชียงใหม่ จำเป็นจะต้องจะต้องจัดทำร่างปฏิทินการศึกษาสำหรับปีการศึกษาถัดไปเป็นประจำทุกปี \enskip
      ซึ่งในการทำร่างปฏิทิน ซึ่งในปฏิทินประกอบไปด้วยกำหนดการของกิจกรรมการศึกษาต่างๆ เช่น วันเปิดภาคเรียน วันลงทะเบียนเรียน วันสอบ เป็นต้น\enskip
      ซึ่งวันกำหนดการต่างๆ จะมีเงื่อนไขที่อ้างอิงกับวันกำหนดการต่างๆ ซึ่งทำให้การทำร่งปฏิทินขึ้นมาหนึ่งร่าง มีความยุ่งยากในการทำมากจนใช้เวลาเป็นสัปดาห์\enskip
      และยังต้องมาแก้ใหม่หากกรรมการของมหาวิทยาลัยเชียงใหม่ไม่ให้ความเห็นชอบ\enskip

      จากปัญหาข้างต้น ผู้จัดทำได้จัดทำโปรแกรมวางแผนปฏิทินการศึกษา มหาวิทยาลัยเชียงใหม่ เพื่อเป็นเครื่องมือที่ช่วยให้ผู้จัดทำร่างปฏิทินของสำนักทะเบียน
      เพื่อช่วยลดระยะในการจัดทำปฏิทินลงได้ อีกทั้งสามารถเพิ่มวามสะดวกในการจัดทำร่างปฏิทินได้ สามารถแก้ไข สร้าง หรือทำซ้ำร่างปฏิทินที่มีฉบับที่คล้าคลึงกัน
      โปรแกรมดังกล่าวก็สามารถตอบสนองได้
   \end{abstractTH}

   \begin{abstract}
      in each academic year Chiang Mai University Registration and Processing Office It is necessary to make
      Outline an academic calendar for the next academic year annually. which in drafting the calendar in which the accompanying calendar
      along with the schedule of various educational activities such as the opening day of the semester registration day, exam date, etc.
      which scheduled dates There will be conditions that refer to various scheduled dates. which makes one draft of the calendar have
      The complexity of making it took weeks. And still have to come and fix it again if the director of the University of Chiang-
      new disapproval
      from the above problem The organizer has prepared an academic calendar planning program. Chiang Mai University for
      A tool that helps drafters of the registry's calendar. To help reduce the time for making calendars, it can also be
      It is convenient to create a draft calendar. You can edit, create or duplicate a draft calendar with similar editions.
      together, such programs can respond.
   \end{abstract}

   \iffalse
      \begin{dedication}
         This document is dedicated to all Chiang Mai University students.

         Dedication page is optional.
      \end{dedication}
   \fi % \iffalse

   \begin{acknowledgments}
           โครงการนี้รับความกรุณาจาก อ.ดร.ชินวัตรอิศราดิสัยกุล อาจารย์ที่ปรึกษาโครงการ
      ที่คอยให้ความช่วยเหลือ เสนอแนะแนวทางในการแก้ปัญหา คอยติดตามงาน และจัดลําดับความสําคัญงานอยู่เสมอ
      รวมไปถึงขอบคุณอาจารย์คณะกรรมการทั้ง อ.ดร. พฤษภ์ บุญมา และ ผศ.ดร. ลัชนา ระมิงค์วงศ์ ที่ให้คำแนะนำต่างๆ รวมไปถึงเพื่อนๆที่ให้คอยให้กำลังใจและช่วยเหลือ ให้คำแนะนำมาตลอด
      ทำให้โครงงานเล่มนี้ออกมาได้อย่างสมบูรณ์ 
      ขอขอบคุณภาควิชาวิศวกรรมคอมพิวเตอร์ คณะวิศวกรรมศาสตร์ มหาวิทยาลัยเชียงใหม่ ที่มอบทุนสนับสนุนการพัฒนา และเอื้อเฟื้อสถานที่ในการจัดทําโครงการนี้
      \acksign{2023}{3}{30}
   \end{acknowledgments}%
\fi % \ifproject

\contentspage

\ifproject
   \figurelistpage % สารบัญรูป 

   \tablelistpage  % สารบัญตาราง
\fi % \ifproject

% \abbrlist % this page is optional

% \symlist % this page is optional

% \preface % this section is optional
