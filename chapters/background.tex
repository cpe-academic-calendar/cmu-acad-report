\chapter{\ifenglish Background Knowledge and Theory\else ทฤษฎีที่เกี่ยวข้อง\fi}

การทำโครงงาน เริ่มต้นด้วยการศึกษาค้นคว้า ทฤษฎีที่เกี่ยวข้อง หรือ งานวิจัย/โครงงาน ที่เคยมีผู้นำเสนอไว้แล้ว ซึ่งเนื้อหาในบทนี้ก็จะเกี่ยวกับการอธิบายถึงสิ่งที่เกี่ยวข้องกับโครงงาน เพื่อให้ผู้อ่านเข้าใจเนื้อหาในบทถัดๆ ไปได้ง่ายขึ้น

% \CIreply{need citation to each of these technologies}
\section{HTML}
HTML~\cite{HTML} ย่อมาจาก HyperText Markup Language เป็น ภาษาคอมพิวเตอร์ท่ีใช้สร้างหน้าเว็บในรูปแบบของไฟล์ HTML (คือไฟล์ที่มีนามสกุลเป็น .htm หรือ .html) ซึ่งมีเว็บเบราว์เซอร์เป็นโปรแกรมที่ใช้แปลงไฟล์ HTML เพื่อ แสดงผลในรูปของหน้าเว็บ
ไฟล์ HTML บันทึกในรูปของไฟล์เอกสาร (text file) ที่สามารถถูกสร้างจากโปรแกรมสร้างไฟล์ ข้อความ เช่น Notepad หรือ word processing ทั่วๆ ไป ซึ่งลักษณะของไฟล์ HTML ประกอบไปด้วยแท็กต่างๆ ที่เป็นคำสั่งของ HTML ซึ่งแท็กจะอยู่ภายในเครื่องหมาย < และ >

\section{CSS}
CSS~\cite{CSS} ย่อมาจาก Cascading Style Sheet  มักเรียกโดยย่อว่า style sheet คือภาษาที่ใช้เป็นส่วนของการจัดรูปแบบการแสดงผลเอกสาร  HTML โดยที่ CSS กำหนดกฏเกณฑ์ในการระบุรูปแบบ (หรือ style) ของเนื้อหาในเอกสาร 
อันได้แก่ สีของข้อความ สีพื้นหลัง ประเภทตัวอักษร และการจัดวางข้อความ ซึ่งการกำหนดรูปแบบนี้ใช้หลักการของการแยกเนื้อหาเอกสาร HTML ออกจากคำสั่งที่ใช้ในการจัดรูปแบบการแสดงผล กำหนดให้รูปแบบของการแสดงผลเอกสาร 
ไม่ขึ้นอยู่กับเนื้อหาของเอกสาร เพื่อให้ง่ายต่อการจัดรูปแบบการแสดงผลลัพธ์ของเอกสาร HTML

\section{JavaScript}
JavaScript~\cite{JavaScript} หรือย่อด้วย JS เป็นภาษาเขียนโปรแกรมที่ถูกพัฒนาและปฏิบัติตามข้อกำหนดมาตรฐานของ ECMAScript
ภาษา JavaScript นั้นเป็นภาษาระดับสูง คอมไพล์ในขณะที่โปรแกรมรัน (JIT) และเป็นภาษาเขียนโปรแกรมแบบหลายกระบวนทัศน์ เช่น การเขียนโปรแกรมเชิงขั้นตอน การเขียนโปรแกรมเชิงวัตถุ หรือการเขียนโปรแกรมแบบ functional 
ภาษา JavaScript มีไวยากรณ์ที่เหมือนกับภาษา C ใช้วงเล็บเพื่อกำหนดบล็อกของคำสั่ง นอกจากนี้ JavaScript ยังเป็นภาษาที่มีประเภทข้อมูลแบบไดนามิกส์ เป็นภาษาแบบ prototype-based และ first-class function
%
ถือว่าเป็นเทคโนโลยีหลักของการพัฒนาเว็บไซต์ มันทำให้หน้าเว็บสามารถตอบโต้กับผู้ใช้ได้โดยที่ไม่จำเป็นต้องรีเฟรชหน้าใหม่ (dynamic website) เว็บไซต์จำนวนมากใช้ภาษา JavaScript 
สำหรับควบคุมการทำงานที่ฝั่ง client-side นั่นทำให้เว็บเบราว์เซอร์ต่างๆ มี JavaScript engine ที่ใช้สำหรับประมวลผลสคริปของภาษา JavaScript ที่รันบนเว็บเบราว์เซอร์

\section{Node.js}
Node.js~\cite{NodeJS} คือสภาพแวดล้อมการทำงานของภาษา JavaScript นอกเว็บเบราว์เซอร์ที่ทำงานด้วย V8 engine กล่าวคือ สามารถใช้ Node.js ในการพัตนาแอพพลิเคชันแบบ command line แอพพลิเคชัน desktop หรือแม้แต่เว็บเซิร์ฟเวอร์ได้ 
โดยที่ Node.js จะมี APIs ที่เราสามารถใช้สำหรับทำงานกับระบบปฏิบัติการ เช่น การรับค่าและการแสดงผล การอ่านเขียนไฟล์ และการทำงานกับเน็ตเวิร์ก เป็นต้น

\section{React.JS}
React~\cite{ReactJS} เป็น JavaScript library ที่ใช้สำหรับสร้าง user interface ในฝั่งด้าน front-end ที่ให้เราสามารถเขียนโค้ดในการสร้าง UI ที่มีความซับซ้อนแบ่งเป็นส่วนเล็กๆ ออกจากกันได้ ซึ่งแต่ละส่วนสามารถแยกการทำงานออกจากกันได้อย่างอิสระ 
และทำให้สามารถนำชิ้นส่วน UI เหล่านั้นไปใช้ซ้ำได้อีก

\section{MongoDB}
MongoDB~\cite{MongoDB} เป็น open-source document database โดยเป็นฐานข้อมูลแบบ NoSQL หรือเรียกว่าไม่มีความสัมพันธ์ของตารางแบบ SQL ทั่วๆไป แต่จะเก็บข้อมูลเป็นแบบ JSON แทน การบันทึกข้อมูลทุกๆ record ใน MongoDB 
เราจะเรียกมันว่า document ซึ่งจะเก็บค่าเป็น key และ value หรือก็คือไฟล์ JSON

\section{JSON}
JSON~\cite{1_JSON} ย่อมาจาก JavaScript Object Notation เป็นฟอร์แมตสำหรับแลกเปลี่ยนข้อมูลคอมพิวเตอร์
ฟอร์แมต JSON นั้นอยู่ในรูปข้อความธรรมดา (plain text) ที่ทั้งมนุษย์และโปรแกรมคอมพิวเตอร์สามารถอ่านเข้าใจได้ 
มีนามสกุลของไฟล์เป็น .json ปัจจุบัน JSON นิยมใช้ในเว็บแอปพลิเคชัน โดย JSON เป็นฟอร์แมตทางเลือกในการส่งข้อมูล นอกเหนือไปจาก XML ซึ่งนิยมใช้กันอยู่แต่เดิม 
สาเหตุที่ JSON เริ่มได้รับความนิยมเป็นเพราะกระชับและเข้าใจง่ายกว่า XML เนื่องจากไฟล์ XML มีหน้าตาเป็นการเก็บข้อมูลโดยเก็บไว้ใน tag
ที่จะต้องมี <tag> เปิดและ </tag> ปิด เหมือนกับ HTML ทำให้การเก็บข้อมูแต่ละตัวต้องใช้พื้นที่มากขึ้น

JSON~\cite{2_JSON} นั้นใช้ในการเก็บข้อมูลโดยมีการรับและส่งไฟล์ด้วยภาษา JavaScript แต่ไม่ถูกมองว่าเป็นภาษาโปรแกรม กลับถูกมองว่าเป็นภาษาในการแลกเปลี่ยนข้อมูลมากกว่า 
ในปัจจุบันมีไลบรารีของภาษาโปรแกรมอื่นๆ ที่ใช้ประมวลผลข้อมูลในรูปแบบ JSON มากมาย

\section{\ifcpe%
ความรู้ตามหลักสูตรซึ่งถูกนำมาใช้หรือบูรณาการในโครงงาน
\else%
ISNE knowledge used, applied, or integrated in this project
\fi
}
\paragraph{261207 Basic CPE Lab}
นำความรู้ทางด้าน HTML, CSS, JavaScript และ React.JS รวมถึง Node.js 
ทั้งด้านของ front-end ที่แสดงผลผ่านเว็บไซต์ และ back-end ที่เชื่อมต่อกับฐานข้อมูล

\paragraph{261342 Database Systems}
การนำข้อมูลมาเก็บในรูปแบบตาราง การทำ UML และการออกแบบ ER diagrams ของเว็บแอพพลิเคชั่นที่จะทำ

\paragraph{261361 Software Engineering}
การใช้กระบวนการทางวิศวกรรมในการดูแลการผลิต ตั้งแต่การเริ่มเก็บความต้องการ การตั้งเป้าหมายของระบบ 
การออกแบบ กระบวนการพัฒนา การตรวจสอบ การประเมินผล และทดสอบระบบ

\paragraph{261102 Computer Programming}
การเขียน ทดสอบ และดูแลซอร์สโค้ดของโปรแกรมคอมพิวเตอร์ ซึ่งซอร์สโค้ดนั้นจะเขียนด้วยภาษาโปรแกรม 
ขั้นตอนการเขียนโปรแกรมต้องการความรู้ในหลายด้านด้วยกัน รวมถึงการนำ GitHub มาใช้ในการเขียนโปรแกรม

\section{\ifcpe%
ความรู้นอกหลักสูตรซึ่งถูกนำมาใช้หรือบูรณาการในโครงงาน
\else%
Extracurricular knowledge used, applied, or integrated in this project
\fi
}
\paragraph{User interface and user experience}
จะทำงานเกี่ยวกับการค้นคว้าหาข้อมูล ทำความเข้าใจถึงความต้องการของผู้ใช้งาน เพื่อให้ทราบถึงปัญหาของผู้ใช้งาน
ออกแบบสินค้าและบริการให้ตรงความต้องการผู้ใช้มากที่สุด ซึ่งหน้าที่หลัก ๆ ของ UX design คือ 
รวบรวมข้อมูลเกี่ยวกับผู้ใช้งาน รวมถึงทำการค้นคว้า ออกแบบ ทดสอบ และประเมินผลสิ่งต่าง ๆ 
ที่เกี่ยวข้องกับผู้ใช้งาน และทำการออกแบบต่าง ๆ เพื่อนำมาประยุกต์กับการออกแบบประสบการณ์ผู้ใช้
 เช่น design thinking, design sprint และ Lean UX
