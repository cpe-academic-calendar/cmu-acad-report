\chapter{\ifenglish Background Knowledge and Theory\else ทฤษฎีที่เกี่ยวข้อง\fi}

การทำโครงงาน เริ่มต้นด้วยการศึกษาค้นคว้า ทฤษฎีที่เกี่ยวข้อง หรือ งานวิจัย/โครงงาน ที่เคยมีผู้นำเสนอไว้แล้ว ซึ่งเนื้อหาในบทนี้ก็จะเกี่ยวกับการอธิบายถึงสิ่งที่เกี่ยวข้องกับโครงงาน เพื่อให้ผู้อ่านเข้าใจเนื้อหาในบทถัดๆ ไปได้ง่ายขึ้น

\section{หน้าเว็บ}
หน้าเว็บ คือ หน้าเอกสารที่ถูกแสดงโดย เว็บเบราว์เซอร์ เพ่ือแสดงข้อมูลต่างๆ ที่เป็นข้อความ รูปภาพ และส่ือผสมต่างๆ ซึ่งเนื้อหาของหน้าเว็บเป็นอย่างไร ขึ้นอยู่กับวัตถุประสงค์ของ เจ้าของหน้าเว็บ ไม่ว่าจะเป็นเนื้อหาเกี่ยวกับการศึกษาธุรกิจ หรือ ความบันเทิง เป็นต้น


\section{ HTML}
HTML ย่อมาจาก HyperText Markup Language เป็น ภาษาคอมพิวเตอร์ท่ีใช้สร้างหน้าเว็บในรูปแบบของ ไฟล์ HTML (คือไฟล์ที่มีนามสกุลเป็น .htm หรือ .html) ซึ่งมีเว็บเบราว์เซอร์เป็นโปรแกรมที่ใช้แปลงไฟล์ HTML เพื่อ แสดงผลในรูปของหน้าเว็บ
ไฟล์ HTML เป็นไฟล์รหัสแอสกี (ASCII) ถูกบันทึกในรูปของไฟล์เอกสาร (Text File) ที่สามารถถูกสร้างจากโปรแกรมสร้างไฟล์ ข้อความ เช่น Notepad หรือ Word Processing ทั่วๆ ไป ซึ่งลักษณะของไฟล์ HTML ประกอบไปด้วยแท็กต่างๆ ที่เป็นคําาส่ังของ HTML ซึ่งแท็กจะอยู่ภายในเครื่องหมาย <และ>

\section{ CSS}
CSS ย่อมาจาก Cascading Style Sheet  มักเรียกโดยย่อว่า "สไตล์ชีต" คือภาษาที่ใช้เป็นส่วนของการจัดรูปแบบการแสดงผลเอกสาร  HTML โดยที่ CSS กำหนดกฏเกณฑ์ในการระบุรูปแบบ (หรือ "Style") ของเนื้อหาในเอกสาร 
อันได้แก่ สีของข้อความ สีพื้นหลัง ประเภทตัวอักษร และการจัดวางข้อความ ซึ่งการกำหนดรูปแบบ หรือ Style นี้ใช้หลักการของการแยกเนื้อหาเอกสาร HTML ออกจากคำสั่งที่ใช้ในการจัดรูปแบบการแสดงผล กำหนดให้รูปแบบของการแสดงผลเอกสาร 
ไม่ขึ้นอยู่กับเนื้อหาของเอกสาร เพื่อให้ง่ายต่อการจัดรูปแบบการแสดงผลลัพธ์ของเอกสาร HTML

\section{JavaScript}
JavaScript หรือย่อด้วย JS เป็นภาษาเขียนโปรแกรมที่ถูกพัฒนาและปฏิบัติตามข้อกำหนดมาตรฐานของ ECMAScript
ภาษา JavaScript นั้นเป็นภาษาระดับสูง คอมไพล์ในขณะที่โปรแกรมรัน (JIT) และเป็นภาษาเขียนโปรแกรมแบบหลายกระบวนทัศน์ เช่น การเขียนโปรแกรมเชิงขั้นตอน การเขียนโปรแกรมเชิงวัตถุ หรือการเขียนโปรแกรมแบบ Functional; 
ภาษา JavaScript มีไวยากรณ์ที่เหมือนกับภาษา C ใช้วงเล็บเพื่อกำหนดบล็อคของคำสั่ง นอกจากนี้ JavaScript ยังเป็นภาษาที่มีประเภทข้อมูลแบบไดนามิกส์ เป็นภาษาแบบ Prototype-based และ First-class function

\section{Node.JS}
Node.js คือสภาพแวดล้อมการทำงานของภาษา JavaScript นอกเว็บเบราว์เซอร์ที่ทำงานด้วย V8 engine นั่นคือสามารถใช้ Node.js ในการพัตนาแอพพลิเคชันแบบ Command line แอพพลิเคชัน Desktop หรือแม้แต่เว็บเซิร์ฟเวอร์ได้ 
โดยที่ Node.js จะมี APIs ที่เราสามารถใช้สำหรับทำงานกับระบบปฏิบัติการ เช่น การรับค่าและการแสดงผล การอ่านเขียนไฟล์ และการทำงานกับเน็ตเวิร์ก เป็นต้น

\section{React.JS}
React เป็น JavaScript library ที่ใช้สำหรับสร้าง user interface ในฝั่งด้าน Front-end ที่ให้เราสามารถเขียนโค้ดในการสร้าง UI ที่มีความซับซ้อนแบ่งเป็นส่วนเล็กๆออกจากกันได้ ซึ่งแต่ละส่วนสามารถแยกการทำงานออกจากกันได้อย่างอิสระ 
และทำให้สามารถนำชิ้นส่วน UI เหล่านั้นไปใช้ซ้ำได้อีก

\section{MongoDB}
MongoDB เป็น open-source document database โดยเป็นฐานข้อมูลแบบ NoSQL หรือเรียกว่าไม่มีความสัมพันธ์ของตารางแบบ SQL ทั่วๆไป แต่จะเก็บข้อมูลเป็นแบบ JSON แทน การบันทึกข้อมูลทุกๆ record ใน MongoDB 
เราจะเรียกมันว่า Documentซึ่งจะเก็บค่าเป็น key และ value หรือก็คือไฟล์ JSON

\section{JSON}
ย่อมาจาก JavaScript Object Notation เป็นฟอร์แมตสำหรับแลกเปลี่ยนข้อมูลคอมพิวเตอร์ ฟอร์แมต JSON นั้นอยู่ในรูปข้อความธรรมดา (plain text) ที่ทั้งมนุษย์และโปรแกรมคอมพิวเตอร์สามารถอ่านเข้าใจได้ 
มาตรฐานของฟอร์แมต JSON คือ RFC 4627 มี Internet media type เป็น application/json และมีนามสกุลของไฟล์เป็น .json
ปัจจุบัน JSON นิยมใช้ในเว็บแอปพลิเคชัน โดยเฉพาะ AJAX โดย JSON เป็นฟอร์แมตทางเลือกในการส่งข้อมูล นอกเหนือไปจาก XML ซึ่งนิยมใช้กันอยู่แต่เดิม สาเหตุที่ JSON เริ่มได้รับความนิยมเป็นเพราะกระชับและเข้าใจง่ายกว่า XML

JSON นั้นใช้ความสัมพันธ์ของภาษาจาวาสคริปต์ แต่ไม่ถูกมองว่าเป็นภาษาโปรแกรม กลับถูกมองว่าเป็นภาษาในการแลกเปลี่ยนข้อมูลมากกว่า ในปัจจุบันมีไลบรารีของภาษาโปรแกรมอื่นๆ ที่ใช้ประมวลผลข้อมูลในรูปแบบ JSON มากมาย







%\section{Third section}
%Section 3 text. The dielectric constant\index{dielectric constant}
%at the air-metal interface determines
%the resonance shift\index{resonance shift} as absorption or capture occurs
%is shown in Equation~\eqref{eq:dielectric}:

%\begin{equation}\label{eq:dielectric}
%k_1=\frac{\omega}{c({1/\varepsilon_m + 1/\varepsilon_i})^{1/2}}=k_2=\frac{\omega
%\sin(\theta)\varepsilon_\mathit{air}^{1/2}}{c}
%\end{equation}

%\noindent
%where $\omega$ is the frequency of the plasmon, $c$ is the speed of
%light, $\varepsilon_m$ is the dielectric constant of the metal,
%$\varepsilon_i$ is the dielectric constant of neighboring insulator,
%and $\varepsilon_\mathit{air}$ is the dielectric constant of air.

%\section{About using figures in your report}

% define a command that produces some filler text, the lorem ipsum.
%\newcommand{\loremipsum}{
%  \textit{Lorem ipsum dolor sit amet, consectetur adipisicing elit, sed do
%  eiusmod tempor incididunt ut labore et dolore magna aliqua. Ut enim ad
%  minim veniam, quis nostrud exercitation ullamco laboris nisi ut
%  aliquip ex ea commodo consequat. Duis aute irure dolor in
%  reprehenderit in voluptate velit esse cillum dolore eu fugiat nulla
%  pariatur. Excepteur sint occaecat cupidatat non proident, sunt in
%  culpa qui officia deserunt mollit anim id est laborum.}\par}

%\begin{figure}
%  \centering

%  \fbox{
%     \parbox{.6\textwidth}{\loremipsum}
%  }

  % To include an image in the figure, say myimage.pdf, you could use
  % the following code. Look up the documentation for the package
  % graphicx for more information.
  % \includegraphics[width=\textwidth]{myimage}

%  \caption[Sample figure]{This figure is a sample containing \gls{lorem ipsum},
%  showing you how you can include figures and glossary in your report.
%  You can specify a shorter caption that will appear in the List of Figures.}
%  \label{fig:sample-figure}
%\end{figure}

%Using \verb.\label. and \verb.\ref. commands allows us to refer to
%figures easily. If we can refer to Figures
%\ref{fig:walrus} and \ref{fig:sample-figure} by name in the {\LaTeX}
%source code, then we will not need to update the code that refers to it
%even if the placement or ordering of the figures changes.

%\loremipsum\loremipsum

% This code demonstrates how to get a landscape table or figure. It
% uses the package lscape to turn everything but the page number into
% landscape orientation. Everything should be included within an
% \afterpage{ .... } to avoid causing a page break too early.
%\afterpage{
%  \begin{landscape}
%  \begin{table}
%    \caption{Sample landscape table}
%    \label{tab:sample-table}

%   \centering

%    \begin{tabular}{c||c|c}
%        Year & A & B \\
%        \hline\hline
%        1989 & 12 & 23 \\
%        1990 & 4 & 9 \\
%        1991 & 3 & 6 \\
%    \end{tabular}
%  \end{table}
%  \end{landscape}
%}

%\loremipsum\loremipsum\loremipsum

%\section{Overfull hbox}

%When the \verb.semifinal. option is passed to the \verb.cpecmu. document class,
%any line that is longer than the line width, i.e., an overfull hbox, will be
%highlighted with a black solid rule:
%\begin{center}
%\begin{minipage}{2em}
%juxtaposition
%\end{minipage}
%\end{center}

%\section{\ifenglish%
%\ifcpe CPE \else ISNE \fi knowledge used, applied, or integrated in this project
%\else%
%ความรู้ตามหลักสูตรซึ่งถูกนำมาใช้หรือบูรณาการในโครงงาน
%\fi
%}

%อธิบายถึงความรู้ และแนวทางการนำความรู้ต่างๆ ที่ได้เรียนตามหลักสูตร ซึ่งถูกนำมาใช้ในโครงงาน

%\section{\ifenglish%
%Extracurricular knowledge used, applied, or integrated in this project
%\else%
%ความรู้นอกหลักสูตรซึ่งถูกนำมาใช้หรือบูรณาการในโครงงาน
%\fi
%}

%อธิบายถึงความรู้ต่างๆ ที่เรียนรู้ด้วยตนเอง และแนวทางการนำความรู้เหล่านั้นมาใช้ในโครงงาน
