\chapter{\ifproject%
\ifenglish Experimentation and Results\else การทดลองและผลลัพธ์\fi
\else%
\ifenglish System Evaluation\else การประเมินระบบ\fi
\fi}
\section{ทดสอบการลงชื่อเข้าใช้} 
   การทดสอบเพื่อ ตรวจสอบว่าผู้ที่เข้าใช้งานเว็บไซต์ เป็นบุคลากรของสำนักทะเบียน มหาวิทยาลัยเชียงใหม่ 
และเมื่อผู้เข้าใช้งานต้องการที่จะเข้าสู่ระบบ log in สามารถเข้าถึงเว็บไซต์ได้ตามปกติ

\section{ทดสอบความแม่นยำและประสิทธิภาพการใช้งานของการวางร่างปฏิทินแบบอัตโนมัติ}
   นำปฏิทินการศึกษาของปีที่ผ่านมามาทดสอบ โดยทดสอบให้ผู้ใช้สร้างปฏิทินการศึกษาโดยนำข้อมูลของปฏิทินจากปีที่ผ่านมา 
เพื่อให้ร่างปฏิทินการศึกษาโดยมีการกำหนดวันกิจกรรมแบบอัตโนมัติ แล้วจึงนำมาเปรียบเทียบกับปฏิทินการศึกษาที่เคยร่างไว้แล้ว 
เมื่อเปรียบเทียบกันแล้วผลลัพธ์ของการเปรียบเทียบที่ได้ปฏิทินการศึกษาที่จัดการโดยอัติโนมัติจะต้องไม่ด้อยไปกว่าปฏิทินการศึกษาที่เคยร่างไว้
โดยผู้ที่เป็นคนตัดสินใจว่าไม่ด้อยกว่านั้นเป็นผู้ที่จะมาใช้งานเว็บไซต์ หรือก็คือบุคลากรของสำนักทะเบียน\CIreply{เทียบกับอะไร?}

\section{การประเมินผลระบบ UX/UI } 
จะมีเกณฑ์หลักๆ ตามหลักของ Product ที่ดีดังนี้
\begin{enumerate}
   \item Usable การใช้ง่าย: โดยการนำเว็บไซต์นี้มาลองให้ผู้ใช้งานใช้และถาม feedback หลังใช้งาน
   \item Equitable การเท่าเทียมกัน: คือการนำเว็บไซต์นี้ไปลองใช้งานในคอมพิวเตอร์หลายๆ ขนาด\CIreply{จำเป็นไหม เพราะอะไร}
   \item Enjoyable: ความพึงพอใจของผู้ใช้มากกว่าการทำปฏิทินการศึกษาด้วยวิธีเดิมที่เคยใช้
   \item Useful มีประโยชน์: เว็บไซต์จะต้องแก้ปัญหาของผู้ใช้ตรงตามจุดประสงค์ของโครงงาน (เป็นการทดสอบการทำงานของระบบด้วย)
\end{enumerate}
\section{}


\CIreply{performance evaluation?}
